%%%%%%%%%%%%%%%%%%%%%%%%%%%%%%%%%%%%%%%%%%%%%%%%%%%%%%%%%%%%%%%%%%%%%%%
% Load packages
\usepackage[hmargin=2cm,vmargin=2.5cm]{geometry} % to control margins: hmargin is left/right margin, and vmargin is top/bottom margin

% The following is used to get ragged right toc
\usepackage{tocbasic}
\DeclareTOCStyleEntry[
  indent=0pt,
  numwidth=0pt,
  pagenumberbox={\mbox},
  raggedentrytext
]{tocline}{part}
\DeclareTOCStyleEntry[
  pagenumberbox={\mbox},
  raggedentrytext
]{tocline}{chapter}
\DeclareTOCStyleEntries[
  raggedentrytext
]{tocline}{section,subsection,subsubsection,paragraph,subparagraph}

\usepackage[headsepline,automark,markcase=noupper]{scrlayer-scrpage}
\renewcommand*\chaptermarkformat{}
\renewcommand*\sectionmarkformat{}
\addtokomafont{pageheadfoot}{\normalfont}
\setlength{\headheight}{\baselineskip}    

\usepackage{changepage} % needed for detection of odd and even pages, and the adjustwidth environment

\usepackage{titlesec} % the toctitles option includes full title in toc regardless of its length (as opposed to the <alternative name> for example \section[<alternative name>]{name}

\usepackage{appendix} % to get nice appendices
\newcommand{\tocseparator}{\vspace{0.5\normalbaselineskip}} % Command used to have slightly larger separation between sectionslist and appendiceslist and referenceslist in toc.
\usepackage{chngcntr} % Defines commands \counterwithin (which sets up a counter to be reset when another is incremented) and \counterwithout (which unsets such a relationship).
\newcommand*\startappendixeqnumbering{%
%    \counterwithout{equation}{part}%
    \counterwithin{equation}{section}%
    \renewcommand*\theequation{\Alph{section}.\arabic{equation}}}
\newcommand*\startnormaleqnumbering{%
    \counterwithout{equation}{section}%
%    \counterwithin{equation}{part}%
    \renewcommand*\theequation{\arabic{equation}}}

\counterwithin{equation}{part}  %Set the default

\usepackage[sectionbib]{chapterbib} % get separate reference lists for each part
\renewcommand\bibname{References} % rename bibliography as references
\usepackage{cite} % get correct ordering on references, i.e. not [4,1,2] but [1,2,4]. Also collaps references: [2,3,4,5,6] -> [2-6]

\usepackage[hidelinks,pdftex,hypertexnames=false]{hyperref} % needed for links
%\hypersetup{colorlinks=true, citecolor=blue, linkcolor=blue, urlcolor=blue} % Color the links

\pdfstringdefDisableCommands{% To resolve warning from "Token not allowed in a PDF string"
  \let\hskip\empty % this causes the warning for \hskip
}

\usepackage{cleveref} % To have easy referencing for figures, equations, ... noabbrev option can be used for full name. Note that cleveref must be loaded after hyperref
\crefname{subsection}{subsection}{subsections}
\crefname{equation}{Eq.}{Eqs.}

\usepackage{ltablex}
\usepackage{amsthm}
\newtheorem{properties}{Properties}
\newtheorem{definition}{Definition}
\newtheorem{proposition}{Proposition}
\newtheorem{conjecture}{Conjecture}
\newtheorem{theorem}{Theorem}
\newtheorem{lemma}{Lemma}
\newtheorem{remark}{Remark}[section]

% skip the chapter counter and reset enumerations for each "part"
\makeatletter
\renewcommand\thefigure{\@arabic\c@figure}
\renewcommand\thetable{\@arabic\c@table}
\renewcommand\theequation{\@arabic\c@equation}
\renewcommand\thesection{\@arabic\c@section}
\renewcommand\thesubsection{\@arabic\c@section.\@arabic\c@subsection}
\renewcommand\thesubsubsection{\@arabic\c@section.\@arabic\c@subsection.\@arabic\c@subsubsection}
\@addtoreset{definition}{part}
\@addtoreset{theorem}{part}
\@addtoreset{proposition}{part}
\@addtoreset{conjecture}{part}
\@addtoreset{properties}{part}
\@addtoreset{lemma}{part}
\@addtoreset{remark}{part}
\@addtoreset{figure}{part}
\@addtoreset{equation}{part}
\@addtoreset{table}{part}
\@addtoreset{footnote}{part}
\makeatother

\newcommand{\pushright}[1]{\ifmeasuring@#1\else\omit\hfill$\displaystyle#1$\fi\ignorespaces}
\newcommand{\pushleft}[1]{\ifmeasuring@#1\else\omit$\displaystyle#1$\hfill\fi\ignorespaces}
\newcommand{\specialcell}[1]{\ifmeasuring@#1\else\omit$\displaystyle#1$\ignorespaces\fi}

% the following three packages are used for filled box around "Paper i" in front part page
\usepackage{mdframed} 
\usepackage{textpos}
\usepackage{calc}

%%%%%%%%%%%%%%%%%%%%%%%%%%%%%%%%%%%%%%%%%%%%%%%%%%%%%%%%%%%%%%%%%%%%%%%%%%%%%%%%%%%%%%%%%%%%%%%%%%%%%%%%%%%%%%%%
\newtoks\paperJournal
\newtoks\publicationStatus
\newtoks\paperAuthors
\newtoks\paperTitle

\makeatletter
\def\title#1{\gdef\@title{#1}}
\let\@title\@empty
\let\authorSep\@empty
\let\eaddressSep\@empty
\def\pAuthors{}
\def\pAddresses{}
\def\pEaddresses{}
\def\author[#1]#2{\g@addto@macro\pAuthors{\authorSep#2\textsuperscript{#1}\def\authorSep{\unskip,\space}}}% 
\def\address[#1]#2{\g@addto@macro\pAddresses{\textsuperscript{#1}\textit{#2}\par}}% 
\def\eaddress[#1]#2{\g@addto@macro\pEaddresses{\eaddressSep\texttt{#2} (#1)\def\eaddressSep{\unskip,\space}}}

\newbox\absbox
\newenvironment{abstract}{\global\setbox\absbox=\vbox\bgroup
  \hsize=\textwidth%
  \noindent\unskip\textbf{Abstract}
\par\medskip\noindent\unskip\ignorespaces}
{\egroup}


\def\sep{\unskip, }%
\newenvironment{pFrontmatter}{
\setcounter{section}{0}
\let\@title\@empty
\let\authorSep\@empty%
\let\eaddressSep\@empty%
\let\pAuthors\@empty%
\let\pAddresses\@empty%
\let\pEaddresses\@empty%
\title{\the\paperTitle}
}{%
	\clearpage\thispagestyle{empty}%
	\begin{center}%
    	\Large\@title\par\vskip18pt
	    \normalsize\pAuthors\par\vskip10pt%
	    \footnotesize\pAddresses\par\vskip36pt%
	    \hrule\vskip12pt%
	    \ifvoid\absbox\else\unvbox\absbox\par\vskip10pt\fi%
    	\hrule\vskip12pt%
    \end{center}%
	\let\thefootnote\relax\footnotetext{\textsuperscript{$\ast$}Corresponding author.\newline\raggedright\textit{Email addresses:\space}\pEaddresses.}%
	\gdef\thefootnote{\arabic{footnote}}%
	\markboth{\the\paperTitle}{}%
}
\makeatother
\newlength{\blackBoxWidth}
\setlength{\blackBoxWidth}{6.4cm}
\assignpagestyle{\part}{empty} % no page number on part title page
\titleformat{\part}[display]{
\setcounter{section}{0}% Let counter for section (and then also subsection) reset for each part
\setcounter{subsection}{0}% Let counter for section (and then also subsection) reset for each part
\centering\normalfont\Huge\bfseries
}
{
	\begin{textblock*}{\blackBoxWidth}(\paperwidth-1in-\hoffset-\oddsidemargin-\blackBoxWidth, 0cm)
	\begin{mdframed}[userdefinedwidth=\blackBoxWidth,linewidth=0pt,innertopmargin=10.25pt,backgroundcolor=black] % 
		\textcolor{white}{\partname}
	\end{mdframed}
	\end{textblock*}
}
{0pt}
{\vspace*{2cm}\LARGE}
[
	\vspace{1cm}
	\normalfont\Large{\the\paperAuthors}        
	\vfill        
	\large{\the\publicationStatus}
]
\titleformat{\chapter}{\normalfont\Huge\bfseries}{}{0pt}{}
\titleformat{\section}[hang]{\bfseries\Large}{\thesection.\quad}{0pt}{}
\titleformat{\subsection}[hang]{\itshape}{\thesubsection.\quad}{0pt}{}
\titleformat{\subsubsection}[hang]{\itshape}{\thesubsubsection.\quad}{0pt}{}

\newcommand{\includePaper}[3]{
\def\tmpOne{#1}%
\def\tmpTwo{#2}%
\ifx\tmpOne\empty
	\renewcommand{\thepart}{Paper~\Roman{part}\texorpdfstring{\quad}{ }--}
	\renewcommand{\partname}{Paper~\Roman{part}}
\else
	\ifx\tmpTwo\empty
		\renewcommand{\thepart}{#1}
	\else
		\renewcommand{\thepart}{#1\texorpdfstring{\quad}{ }--}
	\fi
	\renewcommand{\partname}{#1\vphantom{Pp}} % \vphantom{Pp} added to get correct vertical spacing
\fi
\part{\the\paperTitle}
#3 % contents
\clearpage
\KOMAoptions{manualmark}
\markright{References}
\cleardoublepage
\KOMAoptions{automark}
}

\crefname{appsec}{Appendix}{Appendices}
\newenvironment{inputAppendices}{
\begin{appendices}
\crefalias{section}{appsec}
\addtocontents{toc}{\tocseparator} % add an empty line before appendices in toc
\renewcommand{\thesection}{\Alph{section}}%
\renewcommand{\thesubsection}{\Alph{section}.\arabic{subsection}}%
\renewcommand{\thefigure}{\thesection\arabic{figure}}%   
\startappendixeqnumbering%
}{
\startnormaleqnumbering%
\end{appendices}
}

\usepackage{bookmark} % To resolve "Difference (2) between bookmark levels is greater" - warning
\usepackage{emptypage} % get empty pages on even numbered pages (no footer and header) if initially empty due to \cleardoublepage command
%\usepackage{blindtext} % To create temporary Lorem Ipsum text
