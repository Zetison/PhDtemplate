\section{The incident wave}
\label{Sec1:incidentWave}
The coefficients $F_n^{(1)}$ and $F_n^{(2)}$ in \Cref{Eq1:IncidentWaveConds} may be computed by using the orthogonality property of the Legendre polynomials in \Cref{Eq1:legendreOrthogonality}. In fact, any square integrable function $\Psi(\vartheta)$ on the interval $[0,\PI]$ can be written as (see~\cite[p. 27]{Ihlenburg1998fea})
\begin{equation}
	\Psi(\vartheta) = \sum_{n=0}^\infty \Psi_n \legendre_n(\cos\vartheta)
\end{equation}
where
\begin{equation}\label{Eq1:H_n}
	\Psi_n = \frac{2n+1}{2}\int_0^\PI \Psi(\vartheta) \legendre_n(\cos\vartheta)\sin\vartheta\idiff\vartheta.
\end{equation}
For example, a plane wave traveling along the $x_3$-axis can be expanded as~\cite[10.1.47]{Abramowitz1965hom}
\begin{equation}\label{Eq1:planeWave}
\begin{aligned}
	p_{\mathrm{inc}}(\vec{x},\omega) &= P_{\mathrm{inc}}(\omega)\euler^{\imag k_1 x_3} = P_{\mathrm{inc}}(\omega)\euler^{\imag k_1 r\cos\vartheta}\\
	&= P_{\mathrm{inc}}(\omega)\sum_{n=0}^\infty (2n+1)\imag^n j_n(k_1 r)\legendre_n(\cos\vartheta)
\end{aligned}
\end{equation}
such that
\begin{align}
	F_n^{(1)} = P_{\mathrm{inc}}(\omega) (2n+1)\imag^n j_n(k_1 R_{0,1})\\
	F_n^{(2)} = P_{\mathrm{inc}}(\omega) (2n+1)\imag^n k_1 j_n'(k_1 R_{0,1}).
\end{align}
Another example of an incident wave satisfying the axisymmetry property, is a wave due to a point source located at $\vec{x}_{\mathrm{s}} = -r_{\mathrm{s}}\vec{e}_3$. The incident wave can then be expressed with the fundamental solution of the Helmholtz equation
\begin{equation}
	p_{\mathrm{inc}}(\vec{x},\omega) = P_{\mathrm{inc}}(\omega)\frac{r_{\mathrm{s}}}{|\vec{x}_{\mathrm{s}}-\vec{x}|}\euler^{\imag k_1 |\vec{x}_{\mathrm{s}}-\vec{x}|},\quad |\vec{x}_{\mathrm{s}}-\vec{x}| = \sqrt{r^2+2r_{\mathrm{s}}r\cos\vartheta + r_{\mathrm{s}}^2}.
\end{equation}
By a simple substitution $v=\cos\vartheta$ in \Cref{Eq1:H_n} one gets
\begin{align}\label{Eq1:coeffsG}
\begin{split}
	F_n^{(1)} &= P_{\mathrm{inc}}(\omega)\frac{2n+1}{2}r_{\mathrm{s}}\int_{-1}^1 \frac{\euler^{\imag k_1 q(v)}}{q(v)}\legendre_n(v)\idiff v\\
	F_n^{(2)} &= P_{\mathrm{inc}}(\omega)\frac{2n+1}{2}r_{\mathrm{s}}\int_{-1}^1 \left(R_{0,1}+r_{\mathrm{s}}v\right)\frac{\euler^{\imag k_1 q(v)}}{q^3(v)}\left[\imag k_1 q(v) - 1\right]\legendre_n(v)\idiff v
	\end{split}
\end{align}
where
\begin{equation*}
	q(v) = \sqrt{R_{0,1}^2+2r_{\mathrm{s}}R_{0,1}v+r_{\mathrm{s}}^2}.
\end{equation*}
%Further substitution yields
%\begin{align*}
%	F_n^{(1)} &= P_{\mathrm{inc}}(\omega)\frac{2n+1}{2}\frac{1}{R_{0,1}}\int_{r_{\mathrm{s}}-R_{0,1}}^{r_{\mathrm{s}}+R_{0,1}} \euler^{\imag k_1 q}\legendre_n\left(\frac{1}{2r_{\mathrm{s}}R_{0,1}}\left(R_{0,1}^2-q^2+r_{\mathrm{s}}^2\right)\right)\idiff q\\
%	F_n^{(2)} &= P_{\mathrm{inc}}(\omega)\frac{2n+1}{2}\frac{1}{R_{0,1}}\int_{r_{\mathrm{s}}-R_{0,1}}^{r_{\mathrm{s}}+R_{0,1}} \frac{1}{2R_{0,1}}\left(R_{0,1}^2+q^2-r_{\mathrm{s}}^2\right)\frac{\euler^{\imag k_1 q}}{q^2}\left[\imag k_1 q - 1\right]\legendre_n\left(\frac{1}{2r_{\mathrm{s}}R_{0,1}}\left(R_{0,1}^2-q^2+r_{\mathrm{s}}^2\right)\right)\idiff q
%\end{align*}
%\begin{align*}
%	F_n^{(1)} &= P_{\mathrm{inc}}(\omega)\frac{2n+1}{2}\euler^{\imag k_1 r_{\mathrm{s}}}\int_{-1}^1 \euler^{\imag k_1 R_{0,1} w}\legendre_n\left(\frac{R_{0,1}}{2r_{\mathrm{s}}}\left(1-w^2\right) - w\right)\idiff w\\
%	F_n^{(2)} &= P_{\mathrm{inc}}(\omega)\frac{2n+1}{2}\euler^{\imag k_1 r_{\mathrm{s}}}\int_{-1}^1 \left(\frac{R_{0,1}^2}{2}\left(1+w^2\right)+r_{\mathrm{s}}R_{0,1}w\right)\frac{\euler^{\imag k_1 R_{0,1} w}}{(r_{\mathrm{s}}+R_{0,1}w)^2}\\
%	&{\hskip14em\relax}\cdot\left[\imag k_1 (r_{\mathrm{s}}+R_{0,1}w) - 1\right]\legendre_n\left(\frac{R_{0,1}}{2r_{\mathrm{s}}}\left(1-w^2\right) - w\right)\idiff w
%\end{align*}
One can obtain simple expressions for some of these coefficients, for example
\begin{equation*}
	F_0^{(1)} = P_{\mathrm{inc}}(\omega)\sinc\left(k_1R_{0,1}\right)\euler^{\imag k_1 r_{\mathrm{s}}}.
\end{equation*}
But in general, one needs to use a numerical routine to evaluate the integrals in \Cref{Eq1:coeffsG}.