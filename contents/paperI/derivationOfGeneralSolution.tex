\section{General solution in the solid domain}
\label{Eq1:generalSolution}
It turns out that Navier's equation can be reduced to a set of Helmholtz equations. Since the fluid domain also is governed by the Helmholtz equation, both solid and fluid domains share the same fundamental solutions, and it thus suffices to present the general solution in the solid domain.

\subsection{Lam\'{e} solution}
Fender~\cite{Fender1972sfa} shows that the solution of \Cref{Eq1:navier} can be written in terms of a scalar potential $\phi$ and a vector potential $\vec{\psi}$ as follows
\begin{equation}\label{Eq1:LameSolution}
	\vec{u} = \nabla\phi + \nabla\times\vec{\psi}.
\end{equation}
Such a solution of Navier's equation is called a Lam\'{e} solution. The potentials $\phi$ and $\vec{\psi}$ satisfy the scalar and vector Helmholtz equation, respectively. That is,
\begin{align}
	&\nabla^2\phi + a^2\phi = 0\label{Eq1:phiHelmholtz}\\
	&\nabla^2\vec{\psi} + b^2\vec{\psi} = \zerovec\label{Eq1:PsiHelmholtz}
\end{align}
where
\begin{equation}\label{Eq1:waveNumber_a_b}
	a=\frac{\omega}{c_{\mathrm{s},1}},\quad b=\frac{\omega}{c_{\mathrm{s},2}},\quad c_{\mathrm{s},1} = \sqrt{\frac{3K+4G}{3\rho_{\mathrm{s}}}},\quad c_{\mathrm{s},2} = \sqrt{\frac{G}{\rho_{\mathrm{s}}}}.
\end{equation}
Here, the parameters $c_{\mathrm{s},1}$ and $c_{\mathrm{s},2}$ are the longitudinal and transverse (elastic) wave velocities, respectively, and $a$ and $b$ are the corresponding angular wave numbers in the solid. 

Throughout this work, axisymmetry around the $x_3$-axis is assumed. Assuming symmetry around this particular axis causes no loss of generality, as both the incident wave and the spherical shell share this symmetry property (a simple orthogonal transformation restores the generality of axisymmetry about an arbitrary axis). In the spherical coordinate system, the pressure $p$ and the displacement $\vec{u}$ are then independent of the azimuth angle $\varphi$ in the fluid and solid domains, respectively. Moreover, the solid component in the azimuth angle direction is zero, $u_\upvarphi=0$. This is a result of the axisymmetry of the problem.

\subsection{Series representation using separation of variables}
Using these assumptions Fender~\cite{Fender1972sfa} shows that $\vec{\psi}=\psi_\upvarphi\vec{e}_\upvarphi$, such that when \Cref{Eq1:phiHelmholtz,Eq1:PsiHelmholtz} are expanded in terms of spherical coordinates, the following is obtained (using \Cref{Eq1:laplaceScalarSpherical,Eq1:LapVecPotSpherical})
\begin{align}
	\pderiv{}{r}\left(r^2\pderiv{\phi}{r}\right) + \frac{1}{\sin\vartheta}\pderiv{}{\vartheta}\left(\sin\vartheta\pderiv{\phi}{\vartheta}\right) + (ar)^2\phi &= 0\\
	\pderiv{}{r}\left(r^2\pderiv{\psi_\upvarphi}{r}\right) + \frac{1}{\sin\vartheta}\pderiv{}{\vartheta}\left(\sin\vartheta\pderiv{\psi_\upvarphi}{\vartheta}\right) + \left[(br)^2-\frac{1}{\sin^2\vartheta}\right]\psi_\upvarphi &= 0.
\end{align}
Using separation of variables, each of these equations can be reduced to a couple of spherical Bessel and Legendre equations, with the associate Legendre polynomials of zero and first order (described in \Cref{subsec:legendre}) and spherical Bessel functions (described in \Cref{subsec:sphericalBesselAndHankel}) as solutions. More explicitly,
\begin{align}
	\phi(r,\vartheta) &= \sum_{n=0}^\infty \legendre_n(\cos\vartheta)\left[A_n^{(1)}\besselj_n(ar)+A_n^{(2)}\bessely_n(ar)\right]\label{Eq1:phiSolution}\\
	\psi_\upvarphi(r,\vartheta) &= \sum_{n=0}^\infty \legendre_n^1(\cos\vartheta)\left[B_n^{(1)}\besselj_n(br)+B_n^{(2)}\bessely_n(br)\right]\label{Eq1:psiSolution}
\end{align}
where the coefficients $A_n^{(i)},B_n^{(i)}\in\C$, $i=1,2$, are chosen such that the boundary conditions are satisfied.

By using \Cref{Eq1:LegendreRelation1} these functions and their partial derivatives will have their $\vartheta$-dependency contained in functions of the form (the ones relevant for this work are listed in \Cref{Eq1:Qs})
\begin{equation}
	Q_n^{(j)}(\vartheta) = \deriv[j]{}{\vartheta}\legendre_n(\cos\vartheta).
\end{equation}
That is, there is no need for the associated Legendre polynomials.

For ease of notation, the function $Z_n^{(i)}(\zeta)$, $i=1,2$, is introduced (as in~\cite{Chang1994voa, Chang1994soa}), where 
\begin{equation}
	Z_n^{(1)}(\zeta) = \besselj_n(\zeta),\quad Z_n^{(2)}(\zeta) = \bessely_n(\zeta).
\end{equation}
Moreover, the notation $\xi = \xi(r) = ar$ and $\eta = \eta(r) = br$ is used for convenience. Using the Einstein summation convention, \Cref{Eq1:phiSolution,Eq1:psiSolution} may now be rewritten as
\begin{align}
	\phi(r,\vartheta) &= \sum_{n=0}^\infty Q_n^{(0)}(\vartheta)A_n^{(i)}Z_n^{(i)}(\xi)\label{Eq1:phiSolutionSimplified}\\
	\psi_\upvarphi(r,\vartheta) &= \sum_{n=0}^\infty Q_n^{(1)}(\vartheta)B_n^{(i)}Z_n^{(i)}(\eta).\label{Eq1:psiSolutionSimplified}
\end{align}

\subsection{Expressions for the displacement and stress field}
By expanding \Cref{Eq1:LameSolution} in spherical coordinates (using \Cref{Eq1:delScalarSpherical,Eq1:CrossVecPotSpherical}) yields
\begin{equation}
	\vec{u} = \nabla\phi + \nabla\times\vec{\psi} = \pderiv{\phi}{r}\vec{e}_{\mathrm{r}} + \frac{1}{r}\pderiv{\phi}{\vartheta}\vec{e}_{\upvartheta} + \frac{1}{r\sin\vartheta}\pderiv{}{\vartheta}\left(\psi_\upvarphi\sin\vartheta\right)\vec{e}_{\mathrm{r}} - \frac{1}{r}\pderiv{}{r}\left(r\psi_\upvarphi\right)\vec{e}_{\upvartheta} 
\end{equation}
such that
\begin{equation}
	u_{\mathrm{r}} = \pderiv{\phi}{r} + \frac{1}{r}\pderiv{\psi_\upvarphi}{\vartheta} + \frac{1}{r}\psi_\upvarphi\cot\vartheta
\end{equation}
and
\begin{equation}
	u_\upvartheta = \frac{1}{r}\pderiv{\phi}{\vartheta} -\pderiv{\psi_\upvarphi}{r} -\frac{1}{r}\psi_\upvarphi.
\end{equation}
Insertion of \Cref{Eq1:phiSolutionSimplified,Eq1:psiSolutionSimplified} (using \Cref{Eq1:expandedLegendreEquationIdentity,Eq1:LegendreRelation1,Eq1:BesselDerivIdentity2})
yields
\begin{equation}\label{Eq1:u_rgen}
	u_{\mathrm{r}} = \frac{1}{r}\sum_{n=0}^\infty Q_n^{(0)}(\vartheta)\left[A_n^{(i)}S_{1,n}^{(i)}(\xi)+B_n^{(i)}T_{1,n}^{(i)}(\eta)\right]
\end{equation}
and
\begin{equation}\label{Eq1:u_tgen}
	u_\upvartheta = \frac{1}{r}\sum_{n=0}^\infty Q_n^{(1)}(\vartheta)\left[A_n^{(i)}S_{2,n}^{(i)}(\xi)+B_n^{(i)}T_{2,n}^{(i)}(\eta)\right]
\end{equation}
where
\begin{align*}
	S_{1,n}^{(i)}(\xi) &= \xi \deriv{}{\xi}Z_n^{(i)}(\xi) =  nZ_n^{(i)}(\xi)-\xi Z_{n+1}^{(i)}(\xi)\\ 
	T_{1,n}^{(i)}(\eta) &= -n(n+1)Z_n^{(i)}(\eta)\\
	S_{2,n}^{(i)}(\xi) &= Z_n^{(i)}(\xi)\\
	T_{2,n}^{(i)}(\eta) &= -Z_n^{(i)}(\eta)-\eta \deriv{}{\eta}Z_n^{(i)}(\eta) = -(n+1)Z_n^{(i)}(\eta) + \eta Z_{n+1}^{(i)}(\eta).
\end{align*}
To compute the stresses defined in \Cref{Sec1:sphericalCoordinates}, the partial derivatives of the displacement field in the spherical coordinate system are needed. These derivatives are found to be (using \Cref{Eq1:expandedLegendreEquationIdentity2,Eq1:BesselDerivIdentity1,Eq1:BesselDerivIdentity2})
\begin{align}
	\pderiv{u_{\mathrm{r}}}{r} &= \frac{1}{r^2}\sum_{n=0}^\infty Q_n^{(0)}(\vartheta)\left[A_n^{(i)}S_{3,n}^{(i)}(\xi)+B_n^{(i)}T_{3,n}^{(i)}(\eta)\right]\\
	\pderiv{u_{\upvartheta}}{r} &= \frac{1}{r^2}\sum_{n=0}^\infty Q_n^{(1)}(\vartheta)\left[A_n^{(i)}S_{4,n}^{(i)}(\xi)+B_n^{(i)}T_{4,n}^{(i)}(\eta)\right]\\
	\pderiv{u_{\mathrm{r}}}{\vartheta} &= \frac{1}{r}\sum_{n=0}^\infty Q_n^{(1)}(\vartheta)\left[A_n^{(i)}S_{1,n}^{(i)}(\xi)+B_n^{(i)}T_{1,n}^{(i)}(\eta)\right]\\
	\pderiv{u_\upvartheta}{\vartheta} &= \frac{1}{r}\sum_{n=0}^\infty Q_n^{(2)}(\vartheta)\left[A_n^{(i)}S_{2,n}^{(i)}(\xi)+B_n^{(i)}T_{2,n}^{(i)}(\eta)\right]
\end{align}
where
\begin{align*}
	S_{3,n}^{(i)}(\xi) &= \xi \deriv{}{\xi}S_{1,n}^{(i)}(\xi) - S_{1,n}^{(i)}(\xi) =  (n^2-\xi^2-n)Z_n^{(i)}(\xi) + 2\xi Z_{n+1}^{(i)}(\xi)\\ 
	T_{3,n}^{(i)}(\eta) &= \eta \deriv{}{\eta}T_{1,n}^{(i)}(\eta) -T_{1,n}^{(i)}(\eta) = -n(n+1)\left[(n-1)Z_n^{(i)}(\eta) - \eta Z_{n+1}^{(i)}(\eta)\right]\\
	S_{4,n}^{(i)}(\xi) &= \xi \deriv{}{\xi}Z_n^{(i)}(\xi) - Z_n^{(i)}(\xi) = (n-1)Z_n^{(i)}(\xi)-\xi Z_{n+1}^{(i)}(\xi)\\ 
	T_{4,n}^{(i)}(\eta) &= \eta\deriv{}{\eta}T_{2,n}^{(i)}(\eta) - T_{2,n}^{(i)}(\eta) = (\eta^2-n^2+1)Z_n^{(i)}(\eta) -\eta Z_{n+1}^{(i)}(\eta).
\end{align*}
Using \Cref{Eq1:constitutiveRelationSpherical,Eq1:strainsInSpherical}, and the relation\footnote{This relation is obtained by inserting the definition of the angular wave numbers $a$ and $b$ (\Cref{Eq1:waveNumber_a_b}) into the left-hand side.}
\begin{equation}
	\frac{1}{2}\left(\frac{b}{a}\right)^2 = \frac23+\frac{K}{2G}
\end{equation}
the following formulas for the stress field components are obtained\footnote{One can save some work by observing the similarities between $\sigma_{\upvartheta\upvartheta}$ and $\sigma_{\upvarphi\upvarphi}$
\begin{align*}
	\sigma_{\upvartheta\upvartheta} &= \frac{2}{r}\left(K+\frac{G}{3}\right)u_{\mathrm{r}} + \left(K-\frac{2G}{3}\right)\pderiv{u_{\mathrm{r}}}{r} + \frac{3K-2G}{3r}\left(u_\upvartheta\cot\vartheta + \pderiv{u_\upvartheta}{\vartheta}\right) + \frac{2G}{r}\pderiv{u_\upvartheta}{\vartheta}\\
	\sigma_{\upvarphi\upvarphi} &= \frac{2}{r}\left(K+\frac{G}{3}\right)u_{\mathrm{r}} + \left(K-\frac{2G}{3}\right)\pderiv{u_{\mathrm{r}}}{r} + \frac{3K-2G}{3r}\left(u_\upvartheta\cot\vartheta + \pderiv{u_\upvartheta}{\vartheta}\right) + \frac{2G}{r} u_\upvartheta\cot\vartheta.
\end{align*}}
\begin{align}\label{Eq1:stressFieldComponents1}
	\sigma_{\mathrm{r}\mathrm{r}} &= \frac{2G}{r^2}\sum_{n=0}^\infty Q_n^{(0)}(\vartheta)\left[A_n^{(i)} S_{5,n}^{(i)}(\xi) + B_n^{(i)} T_{5,n}^{(i)}(\eta)\right]\\\label{Eq1:stressFieldComponents4}
	\sigma_{\upvartheta\upvarphi} &= 0\\\label{Eq1:stressFieldComponents5}
	\sigma_{\mathrm{r}\upvarphi} &= 0
\end{align}
\begin{align}
	\sigma_{\upvartheta\upvartheta} &= \frac{2G}{r^2}\sum_{n=0}^\infty\left\{Q_n^{(0)}(\vartheta)\left[A_n^{(i)} S_{6,n}^{(i)}(\xi) + B_n^{(i)} T_{6,n}^{(i)}(\eta)\right]\right.\nonumber\\ \label{Eq1:stressFieldComponents2}
	&{\hskip5em\relax}\left.+  Q_n^{(2)}(\vartheta)\left[A_n^{(i)} S_{2,n}^{(i)}(\xi) + B_n^{(i)} T_{2,n}^{(i)}(\eta)\right]\right\}\\
	\sigma_{\upvarphi\upvarphi} &= \frac{2G}{r^2}\sum_{n=0}^\infty\left\{Q_n^{(0)}(\vartheta)\left[A_n^{(i)} S_{6,n}^{(i)}(\xi) + B_n^{(i)} T_{6,n}^{(i)}(\eta)\right]\right.\nonumber\\ \label{Eq1:stressFieldComponents3}
	&{\hskip5em\relax}\left. +   Q_n^{(1)}(\vartheta)\cot(\vartheta)\left[A_n^{(i)} S_{2,n}^{(i)}(\xi) + B_n^{(i)} T_{2,n}^{(i)}(\eta)\right]\right\}\\\label{Eq1:stressFieldComponents6}
	\sigma_{\mathrm{r}\upvartheta} &= \frac{2G}{r^2}\sum_{n=0}^\infty Q_n^{(1)}(\vartheta)\left[A_n^{(i)} S_{7,n}^{(i)}(\xi) + B_n^{(i)} T_{7,n}^{(i)}(\eta)\right]
\end{align}
where
\begin{align}\label{Eq1:stress}
\begin{split}
	S_{5,n}^{(i)}(\xi) &= \frac{1}{2G}\left[\left(K+\frac{4G}{3}\right)S_{3,n}^{(i)}(\xi) - \left(K-\frac{2G}{3}\right) n(n+1)Z_n^{(i)}(\xi)\right.\\ 
	&{\hskip4em\relax}\left. + 2\left(K-\frac{2G}{3}\right) S_{1,n}^{(i)}(\xi)\right] \\
	&= \left[n^2-n-\frac{1}{2}\left(\frac{b}{a}\right)^2\xi^2\right] Z_n^{(i)}(\xi) + 2\xi Z_{n+1}^{(i)}(\xi)\\
	T_{5,n}^{(i)}(\eta) &= \frac{1}{2G}\left[\left(K+\frac{4G}{3}\right)T_{3,n}^{(i)}(\eta) - \left(K-\frac{2G}{3}\right) n(n+1)T_{2,n}^{(i)}(\eta)\right.\\ 
	&{\hskip4em\relax}\left. + 2\left(K-\frac{2G}{3}\right) T_{1,n}^{(i)}(\eta)\right] \\
	&= -n(n+1)\left[(n-1)Z_n^{(i)}(\eta) - \eta Z_{n+1}^{(i)}(\eta)\right]\\
	S_{6,n}^{(i)}(\xi) &= -\left(\frac{K}{2G}-\frac13\right)n(n+1)S_{2,n}^{(i)}(\xi) +\left(\frac13+\frac{K}{G} \right)S_{1,n}^{(i)}(\xi) + \left(\frac{K}{2G}-\frac13\right)S_{3,n}^{(i)}(\xi)\\
	&= \left[n-\frac{1}{2}\left(\frac{b}{a}\right)^2\xi^2+\xi^2\right] Z_n^{(i)}(\xi) - \xi Z_{n+1}^{(i)}(\xi)\\
	T_{6,n}^{(i)}(\eta) &=  -\left(\frac{K}{2G}-\frac13\right)n(n+1)T_{2,n}^{(i)}(\eta) +\left(\frac13+\frac{K}{G} \right)T_{1,n}^{(i)}(\eta) + \left(\frac{K}{2G}-\frac13\right)T_{3,n}^{(i)}(\eta)\\
	&= -n(n+1)Z_n^{(i)}(\eta)\\
	S_{7,n}^{(i)}(\xi) &= \frac{1}{2}\left[S_{1,n}^{(i)}(\xi) + S_{4,n}^{(i)}(\xi) - S_{2,n}^{(i)}(\xi) \right] \\
	&= (n-1)Z_n^{(i)}(\xi) -\xi Z_{n+1}^{(i)}(\xi)\\
	T_{7,n}^{(i)}(\eta) &= \frac{1}{2}\left[T_{1,n}^{(i)}(\eta) + T_{4,n}^{(i)}(\eta) - T_{2,n}^{(i)}(\eta) \right] \\
	&= -\left(n^2-1-\frac{1}{2}\eta^2\right)Z_n^{(i)}(\eta) - \eta Z_{n+1}^{(i)}(\xi).
	\end{split}
\end{align}
%
\subsection{Validation of the displacement and stress formulas}
The correctness of the formulas may be controlled by considering Navier's equation (\Cref{Eq1:navier}) in spherical coordinates. The three components of Navier's equation in spherical coordinates are given in \Cref{Eq1:navierSpherical1,Eq1:navierSpherical2,Eq1:navierSpherical3}, the last of which is automatically satisfied due to the symmetry assumptions. The first two equations simplify to
\begin{equation}\label{Eq1:navierSphericalSimplified1}%
\pderiv{\sigma_{\mathrm{rr}}}{r} + \frac{1}{r}\pderiv{\sigma_{\mathrm{r}\upvartheta}}{\vartheta} + \frac{1}{r}\left(2\sigma_{\mathrm{r}\mathrm{r}} - \sigma_{\upvartheta\upvartheta} - \sigma_{\upvarphi\upvarphi} + \sigma_{\mathrm{r}\upvartheta}\cot\vartheta\right) +\omega^2\rho_{\mathrm{s}}u_{\mathrm{r}} = 0
\end{equation}
and
\begin{equation}\label{Eq1:navierSphericalSimplified2}
\pderiv{\sigma_{\mathrm{r}\upvartheta}}{r} + \frac{1}{r}\pderiv{\sigma_{\upvartheta\upvartheta}}{\vartheta} + \frac{1}{r}\left[(\sigma_{\upvartheta\upvartheta} - \sigma_{\upvarphi\upvarphi})\cot\vartheta + 3\sigma_{\mathrm{r}\upvartheta} \right] +\omega^2\rho_{\mathrm{s}}u_\upvartheta = 0.
\end{equation}
Differentiation of the stress field components yields
\begin{align*}
	\pderiv{\sigma_{\mathrm{r}\mathrm{r}}}{r} &= \frac{2G}{r^3}\sum_{n=0}^\infty Q_n^{(0)}(\vartheta)\left[A_n^{(i)} S_{8,n}^{(i)}(\xi) + B_n^{(i)} T_{8,n}^{(i)}(\eta)\right]\\
	\pderiv{\sigma_{\upvartheta\upvartheta}}{\vartheta} &= \frac{2G}{r^2}\sum_{n=0}^\infty\left\{Q_n^{(1)}(\vartheta)\left[A_n^{(i)} S_{6,n}^{(i)}(\xi) + B_n^{(i)} T_{6,n}^{(i)}(\eta)\right] \right.\\ 
	&{\hskip5em\relax}\left. +  Q_n^{(3)}(\vartheta)\left[A_n^{(i)} S_{2,n}^{(i)}(\xi) + B_n^{(i)} T_{2,n}^{(i)}(\eta)\right]\right\}\\
	\pderiv{\sigma_{\mathrm{r}\upvartheta}}{r} &= \frac{2G}{r^3}\sum_{n=0}^\infty Q_n^{(1)}(\vartheta)\left[A_n^{(i)} S_{9,n}^{(i)}(\xi) + B_n^{(i)} T_{9,n}^{(i)}(\eta)\right]\\
	\pderiv{\sigma_{\mathrm{r}\upvartheta}}{\vartheta} &= \frac{2G}{r^2}\sum_{n=0}^\infty Q_n^{(2)}(\vartheta)\left[A_n^{(i)} S_{7,n}^{(i)}(\xi) + B_n^{(i)} T_{7,n}^{(i)}(\eta)\right]
\end{align*}
where
\begin{align*}
	S_{8,n}^{(i)}(\xi) &= -2S_{5,n}^{(i)}(\xi) + \xi \deriv{}{\xi}S_{5,n}^{(i)}(\xi) \\
	&= \left[n^3-3n^2+2n-\frac{n}{2}\left(\frac{b}{a}\right)^2\xi^2 + 2\xi^2\right]Z_n^{(i)}(\xi)\\ 
	&{\hskip1em\relax} + \left[-n^2-n-6+\frac{1}{2}\left(\frac{b}{a}\right)^2\xi^2\right]\xi Z_{n+1}^{(i)}(\xi)\\
	T_{8,n}^{(i)}(\eta) &= -2T_{5,n}^{(i)}(\eta) + \eta \deriv{}{\eta}T_{5,n}^{(i)}(\eta) \\
	&= n(n+1)\left[\left(-n^2+3n-2+\eta^2\right)Z_n^{(i)}(\eta) - 4\eta Z_{n+1}^{(i)}(\eta)\right]\\
	S_{9,n}^{(i)}(\xi) &= -2S_{7,n}^{(i)}(\xi) + \xi \deriv{}{\xi}S_{7,n}^{(i)}(\xi) \\
	&= \left[n^2-3n+2-\xi^2\right]Z_n^{(i)}(\xi) + 4\xi Z_{n+1}^{(i)}(\xi) \\
\end{align*}
\begin{align*}
	T_{9,n}^{(i)}(\eta) &= -2T_{7,n}^{(i)}(\eta) + \eta \deriv{}{\eta}T_{7,n}^{(i)}(\eta) \\
	&= \left(-n^3+2n^2+n-2+\frac{n}{2}\eta^2-\eta^2\right)Z_n^{(i)}(\eta)\\ 
	&{\hskip1em\relax} +\left(n^2+n+2-\frac{1}{2}\eta^2\right)\eta Z_{n+1}^{(i)}(\eta).
\end{align*}
Inserting these expressions (alongside the stress components in \Cref{Eq1:stressFieldComponents1,Eq1:stressFieldComponents2,Eq1:stressFieldComponents3,Eq1:stressFieldComponents4,Eq1:stressFieldComponents5,Eq1:stressFieldComponents6}) into \Cref{Eq1:navierSphericalSimplified1,Eq1:navierSphericalSimplified2} and using \Cref{Eq1:expandedLegendreEquationIdentity2,Eq1:expandedLegendreEquationIdentity3}, and observing that
\begin{align*}
	\pderiv{\sigma_{\upvartheta\upvartheta}}{\vartheta} +(\sigma_{\upvartheta\upvartheta} - \sigma_{\upvarphi\upvarphi})\cot\vartheta= \frac{2G}{r^2}&\sum_{n=0}^\infty  Q_n^{(1)}(\vartheta)\left\{A_n^{(i)} S_{6,n}^{(i)}(\xi) + B_n^{(i)} T_{6,n}^{(i)}(\eta)\right. \\
	&{\hskip1em\relax}+\left. \left(-n^2-n+1\right)\left[A_n^{(i)} S_{2,n}^{(i)}(\xi) + B_n^{(i)} T_{2,n}^{(i)}(\eta)\right]\right\},
\end{align*}
the left-hand side of \Cref{Eq1:navierSphericalSimplified1} and \Cref{Eq1:navierSphericalSimplified2} are indeed equal to zero.
