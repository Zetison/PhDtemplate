\chapter*{Abbreviations}
\addcontentsline{toc}{chapter}{Abbreviations}
\markboth{Abbreviations}{Abbreviations}%
\begin{tabularx}{\linewidth}{p{0.17\linewidth} X}
ABC					& Absorbing boundary condition\\
AMR					& Adaptive mesh refinement\\
ANSYS				& Analysis Systems\\
ASCII				& American standard code for information interchange\\
ASI					& Acoustic structure interaction\\
ASIGA				& Acoustic scattering using isogeometric analysis (toolbox)\\
BA					& Best approximation\\
BEM					& Boundary element method\\
BeTSSi				& Benchmark target strength simulation\\
BGC					& Bubnov--Galerkin conjugated\\
BGU					& Bubnov--Galerkin unconjugated\\
BiCGstab			& Biconjugate gradient stabilized method\\
BIE					& Boundary integral equation\\
(C/G)CBIE			& (Collocation/Galerkin) Conventional BIE\\
(C/G)HBIE			& (Collocation/Galerkin) Hypersingular BIE\\
(C/G)BM				& (Collocation/Galerkin) Burton--Miller\\
CAD					& Computer aided design\\
CPU					& Central processing unit\\
CHIEF				& Combined Helmholtz integral formulation\\
DRDC 				& Defence Research and Development Canada\\
ESBC				& Elastic sphere boundary condition\\
FDM					& Finite difference method\\
FEA					& Finite element analysis\\
FEM					& Finite element method\\
FFT					& Fast Fourier Transform\\
FWG					& Die Forschungsanstalt der Bundeswehr f\"{u}r Wasserschall und Geophysik\\
GF					& Gauss--Freud (quadrature)\\
GGL					& Generalized Gauss--Laguerre (quadrature)\\
GL					& Gauss-Legendre (quadrature)\\
GLL					& Gauss-Lobatto-Legendre (nodes)\\
GMRES				& Generalized minimal residual method\\
IE(M)				& Infinite element (method)\\
IENSG				& Infinite element for non-separable geometries\\
IGA					& Isogeometric analysis\\
KDT					& Kirchhoff diffraction theory\\
LR B-splines		& Locally refined B-splines\\
LR 					& Lloyd's Register\\
MATLAB				& MATrix LABoratory\\
MFS					& Method of fundamental solutions\\
NACA				& National Advisory Committee for Aeronautics\\
NNBC				& Neumann-Neumann boundary condition\\
NURBS				& Non-uniform rational B-splines\\
NSD					& Numerical steepest descent\\
ODE					& Ordinary differential equation\\
PDE					& Partial differential equation\\
PGC					& Petrov--Galerkin conjugated\\
PGU					& Petrov--Galerkin unconjugated\\
PML					& Perfectly matched layer\\
RAFAEL 				& Rafael advanced defense systems\\
RAM					& Random-access memory\\
RDDC 				& Recherche et d\'{e}veloppement pour la d\'{e}fense Canada\\
SEM					& Spectral element method\\
SHBC				& Sound-hard boundary condition\\
SSBC				& Sound-soft boundary condition\\
TKMS 				& ThyssenKrupp Marine Systems\\
TNO 				& The Netherlands Organisation for applied scientific research\\
TS					& Target strength\\
WTD 71				& Wehrtechnische Dienststelle f\"{u}r Schiffe und Marinewaffen\\
XIBEM				& Extended isogeometric boundary element method\\
\end{tabularx}

\chapter*{Notation}
\addcontentsline{toc}{chapter}{Notation}
\markboth{Notation}{Notation}%
This thesis follows the following general notation conventions (inspired by the ISO-80000-2 standard).
\begin{itemize}
	\item The usage of bars always indicates complex conjugation. For example, $\bar{p}$.
	\item Indices are denoted by $i,j,l,m,n$.
	\item The usage of breve indicates the time-dependent equivalent function of the corresponding function in the frequency domain. For example, $\breve{p}(\vec{x},t)$, $p(\vec{x},\omega)$.
	\item The usage of marks always indicates derivative. For example, $f'(\xi) = \deriv{f}{\xi}$.
	\item The usage of tildes indicates an alternative variable. For example, $\sigma_{ij}$, $\tilde{\sigma}_{ij}$.
	\item The usage of hats over vectors indicates normalized vectors. For example, $\hat{\vec{x}}=\vec{x}/|\vec{x}|$.
	\item Bold notation is used for vectors, matrices and tensors. For example, $\vec{x}$, $\vec{C}$ and $\vec{\sigma}$. The components should not be written in bold notation, i.e. $x_i$, $C_{ij}$, $\sigma_{ij}$.
	\item Units, fundamental constants and fundamental functions should be in upright mode (Euler's number $\euler$, kilograms $\si{kg}$, meter $\si{m}$, spherical Bessel function $\besselj_n$, spherical Hankel function $\hankel_n$, etc.).
	\item Descriptive text is written in right mode i.e. $p_{\mathrm{inc}}$ (the \underline{inc}ident pressure field).
\end{itemize}

\section*{General notation}
\begin{tabularx}{\linewidth}{p{0.12\linewidth} X}
$\vec{A}$				& Global matrix of linear equations\\
$B_\epsilon(\vec{x})$	& Ball of radius $\epsilon$ centered at $\vec{x}$\\
$B(p,q)$				& Bilinear form\\
$B_{i,\check{p},\vec{t}}$ & The $i^{\mathrm{th}}$ B-spline basis function of degree $\check{p}$ and with knot vector $\vec{t}$\\
$B_n$					& Radial integral in infinite elements\\
$C^n$ 					& Space of continuous functions that have continuous first $n$ derivatives\\
$C_n$ 					& Coefficients for manufactured solution\\
$\C$ 					& Space of complex numbers\\
$c_{\mathrm{f}}$ 		& Sound speed in fluid\\
$\mathrm{d}$ 			& Differential \\
$\vec{d}_{\mathrm{inc}}$& Unit vector pointing in the direction of the incident plane wave\\
$d$ 					& Spatial dimension\\
$D_{m\tilde{m}}$ 		& Coefficients for radial basis functions in the solution space of the infinite elements\\
$\tilde{D}_{n\tilde{n}}$& Coefficients for radial basis functions in the test space of the infinite elements\\
$\vec{e}$ 				& Unit basis vector\\
$\vec{e}_i$ 			& Standard Cartesian basis vector\\
$\vec{e}_{\mathrm{r}}$, $\vec{e}_\upvartheta$, $\vec{e}_\upvarphi$ & Basis vectors for the spherical coordinate system\\
$\euler$ 				& Euler's number $\euler = \num{2.718281828459045}\dots$\\
$E$ 					& Young's modulus\\ 
$E_n$ 					& Exponential integral\\
$f$ 					& Frequency\\
$\fourier$ 				& Fourier transform\\
$g$ 					& Neumann data\\
$G$ 					& Shear modulus\\
$G^n$ 					& Geometric continuity of order $n$ ($G^0$ being a geometry with kinks)\\
$h$, $h_{\mathrm{max}}$  & Maximal element size\\
%$H^1$ 					& Sobolev space\\
$\vec{H}_n$ 			& Global matrix for the linear system of equations for the $n^{\mathrm{th}}$ mode\\
$\hankel_n^{(i)}$ 		& The $n^{\mathrm{th}}$ spherical Hankel function of $i^{\mathrm{th}}$ kind \\
$\imag$ 				& Imaginary unit $\imag = \sqrt{-1}$\\
$\vec{J}$				& Jacobian matrix\\
$J$						& Jacobian determinant\\
$\besselj_n$ 			& The $n^{\mathrm{th}}$ spherical Bessel function of first kind\\
$\besselJ_n$ 			& The $n^{\mathrm{th}}$ Bessel function of first kind\\
$k$ $(k_m)$				& Angular wave number (in the $m^{\mathrm{th}}$ fluid layer)\\
$\check{k}$				& Continuity\\
$K$ 					& Bulk modulus\\
$\vec{K}$				& Stiffness matrix\\
$L(p)$					& Linear form\\
$\vec{M}$				& Mass matrix\\
$\mathcal{M}_{m,\check{p},\check{k}}^{\textsc{IGA}}$ 	& IGA mesh number $m$ with polynomial order $\check{p}$ and continuity $\check{k}$\\
$\mathcal{M}_{m,\check{p},\mathrm{i}}^{\textsc{FEM}}$ 	& Isoparametric FEM mesh number $m$ with polynomial order $\check{p}$\\
$\mathcal{M}_{m,\check{p},\mathrm{s}}^{\textsc{FEM}}$ 	& Subparametric FEM mesh number $m$ with polynomial order $\check{p}$\\
$\N$ 					& Space of non-negative integers\\
$\N^*$ 					& Space of positive integers\\
$N$ 					& Number of basis functions in the radial direction for infinite elements\\
$n_\upxi$ 				& Number of basis functions in the $\xi$-direction\\
$n_\upeta$ 				& Number of basis functions in the $\eta$-direction\\
$n_\upzeta$ 			& Number of basis functions in the $\zeta$-direction\\
$n_\mathrm{dof}$ 		& Number of degrees of freedom\\
$\vec{n}$ 				& Outward pointing (unit) normal vector\\
$\legendre_n$ 			& Legendre polynomial\\
$\legendre_n^m$ 		& Associated Legendre functions\\
$p_0$ 					& Far field pattern\\
$p$, $\breve{p}$ 		& Scattered pressure field\\
$p_{\mathrm{inc}}$ 		& The incident wave\\
$P_{\mathrm{inc}}$ 		& Amplitude of incident wave\\
$p_{\mathrm{tot}}$ 		& Total pressure field $p_{\mathrm{tot}} = p + p_{\mathrm{inc}}$\\
$\check{p}$ 			& Polynomial degree (or degree of NURBS functions)\\
$\check{p}_\upxi$ 		& Degree of basis functions in the $\xi$-direction\\
$\check{p}_\upeta$ 		& Degree of basis functions in the $\eta$-direction\\
$\check{p}_\upzeta$ 	& Degree of basis functions in the $\zeta$-direction\\
$r$ 					& Radius in the spherical (or prolate spheroidal) coordinate system\\
$r_{\mathrm{a}}$ 		& Radius at artificial boundary in the spherical (or prolate spheroidal) coordinate system\\
$\R$ 					& Space of real numbers\\
$R$ 					& Distance between the arguments of $\Phi_k$, $R = |\vec{x}-\vec{y}|$\\
$R_i$ 					& NURBS basis function (with global index $i$)\\
$R_{i,j,l}$ 			& NURBS basis function (with local indices $i$, $j$ and $l$)\\
$\mathcal{S}$ 			& Solution space\\
$\mathcal{S}_h$ 		& Finite dimensional solution space\\
$t$ 					& Time variable\\
$\Xi$ 					& Knot vector in the $\xi$-direction\\
$\Eta$ 					& Knot vector in the $\eta$-direction\\
$\Zeta$ 				& Knot vector in the $\zeta$-direction\\
$T$ 					& Period\\
$\vec{T}$ 				& Exterior traction vector\\
$\TS$ 					& Target strength\\
$\vec{x}$ 				& Spatial variable in Cartesian coordinates\\
$\vec{X}$ 				& Geometric parameterization\\
$\vec{u}$, $\breve{\vec{u}}$ & Displacement in the solid domain\\
$\mathcal{V}$ 			& Test space\\
$\mathcal{V}_h$ 		& Finite dimensional test space\\
$\bessely_n$ 			& The $n^{\mathrm{th}}$ spherical Bessel function of second kind\\
$\besselY_n$ 			& The $n^{\mathrm{th}}$ Bessel function of second kind\\
$\vec{y}_n$ 			& Source points for manufactured solution\\
$\Z$ 					& Space of integers\\
$\alpha$ ($\alpha_{\mathrm{s}}$) 			& Aspect angle (for the source point)\\
$\Gamma$				& Boundary of scatterer\\
$\Gamma_{\check{p}}$	& Outer boundary of BeTSSi submarine ($\check{p}\geq 2$)\\
$\upgamma$ 				& Euler-Mascheroni constant $\upgamma = \num{0.577215664901532861}\dots$\\
$\delta_{ij}$ 			& Kronecker delta\\
$\Delta$ 				& Triangle constructor $\Delta(\vec{x}_1,\vec{x}_2,\vec{x}_3)\subset\R^d$\\
$\upDelta$ 				& Difference operator\\
$\varepsilon$ 			& Machine epsilon precision\\
$\varepsilon_{ij}$ 		& Strain field in Cartesian coordinates\\
$\eta$ 					& Second parameter of the parameter space\\
$\zeta$ 				& Third parametric NURBS parameter\\
$\vartheta$ ($\vartheta_{\mathrm{s}}$) 			& Polar angle in the spherical coordinate system (at the source point)\\
$\boldsymbol\kappa$ 	& Set of indices (for NURBS functions in solution/test space)\\
$\lambda$ 				& Wavelength\\
$\nu$ 					& Poisson's ratio\\
$\xi$ 					& First parameter of the parameter space\\
$\xi_1$, $\xi_2$, $\xi_3$ & Area coordinates (barycentric coordinates) of a triangle\\
$\PI$ 					& Archimedes' constant (pi) $\PI = \num{3.141592653589793}\dots$\\
$\rho$ 					& Substitution variable $\varrho=r/r_{\mathrm{a}}$\\
$\varrho_1$ 			& Simplifying notation variable $\varrho_1=\Upsilon/r_{\mathrm{a}}$\\
$\varrho_2$ 			& Simplifying notation variable $\varrho_2=kr_{\mathrm{a}}$\\
$\varrho_3$ 			& Simplifying notation variable $\varrho_3=k\Upsilon$\\
$\rho_{\mathrm{f}}$ 	& Mass density of fluid\\
$\rho_{\mathrm{s}}$ 	& Mass density of solid\\
$\sigma_{ij}$ 			& Stress field in Cartesian coordinates\\
$\tau$ 					& Minimal number of degrees of freedom per wavelength\\
$\Upsilon$ 				& Focus in the elliptic/prolate spheroidal coordinate system\\
$\varphi$ ($\varphi_{\mathrm{s}}$)	& Azimuth angle in the spherical coordinate system (at the source point)\\
$\Phi_k$ 				& Fundamental solution of Helmholtz equation\\
$\omega$ 				& Angular frequency\\
$\Omega_{\mathrm{s}} $ 	& Solid domain\\
$\Omega^+$ 				& Unbounded exterior fluid domain\\
$\Omega^-$ 				& Interior fluid domain\\
$\partial$ 				& Partial derivatives \\
%$\nabla$ 				& Nabla operator\\
%$\nabla^2$ 				& Laplace operator\\
%$\prod$ 				& Product notation\\
%$\sum$	 				& Sum notation\\
\end{tabularx}

\section*{Notation for paper I}
\begin{tabularx}{\linewidth}{p{0.12\linewidth} X}
$a$ ($a_m$) 			& Longitudinal angular wave number (in the $m^{\mathrm{th}}$ shell)\\
$A_n^{(i)}$ 			& Coefficients of potential function $\phi$ in the solid domain\\
$b$ ($b_m$) 			& Transverse angular wave number (in the $m^{\mathrm{th}}$ shell)\\
$B_n^{(i)}$ 			& Coefficients of potential function $\psi_\upvarphi$ in the solid domain\\
$B$ 					& Bandwidth\\
$\vec{B}_{m,n}$ 		& Local vector of unknown coefficients in the linear system of equations for the $n^{\mathrm{th}}$ mode\\
$c_{\mathrm{s},1}$ 		& The longitudinal wave velocity in a solid\\
$c_{\mathrm{s},2}$ 		& The transverse wave velocity in a solid\\
$C_\varepsilon$ 		& Upper bound for scaled frequency, regarding round-off error with precision $\varepsilon$\\
$\vec{C}$ 				& Linear elasticity matrix\\
$\vec{C}_n$ 			& Vector of unknown coefficients in the system of equation for the $n^{\mathrm{th}}$ mode\\
$\vec{C}_{m,n}$ 		& Local vector of unknown coefficients in the system of equations for the $n^{\mathrm{th}}$ mode\\
$C_n^{(i)}$ 			& Coefficients of the scattered pressure $p$\\
$\vec{D}$ 				& Stress coordinate transformation matrix\\
$\vec{D}_n$ 			& Right hand side vector for the linear system of equation for the  $n^{\mathrm{th}}$ mode\\
$f_{\mathrm{c}}$ 		& Center frequency in wavelet\\
$F_n^{(i)}$ 			& Coefficients of the incident wave expanded in Legendre functions at the outer surface\\
$G_{p,q}^{m,n}$ 		& Meijer G-function\\
$h_{\mathrm{r}}$, $h_\upvartheta$, $h_\upvarphi$ & Scale factors for the spherical (or prolate spheroidal) coordinate system\\
$\vec{H}_{m,n}^{(i,j)}$ & Submatrices of $\vec{H}_n$\\
$M$ 					& Number of spherical shells\\
$N$ 					& Number of terms in Fourier sum\\
$N_\varepsilon$ 		& Infinite series truncation number\\
$Q_n^{(i)}$ 			& Derivatives of Legendre polynomials with cosine argument\\
$R_{0,m}$ 				& Outer surface radius of $m^{\mathrm{th}}$ spherical shell\\
$R_{1,m}$ 				& Outer surface radius of $m^{\mathrm{th}}$ spherical shell\\
$S_{j,n}^{(i)}$ 		& Simplifying notation. Superposition of Bessel functions with argument $\xi = ar$\\
$T_{j,n}^{(i)}$ 		& Simplifying notation. Superposition of Bessel functions with argument $\eta = br$\\
$Z_n^{(i)}$ 			& Spherical Bessel functions\\
$\alpha_{ij}$ 			& Cosine between two basis vectors in a coordinate transformation\\
%$\epsilon$ 			&  \\
$\zeta$ 				& Simplifying notation $\zeta=kr$\\
$\eta$ 					& Simplifying notation $\eta=br$\\
%$\Theta$ 				& \\
%$\vartheta$ 			& \\
%$\iota$ 				& \\
%$\kappa$ 				& \\
%$\mu$ 					& \\
$\xi$ 					& Simplifying notation $\xi=ar$\\
%$\Xi$ 					& \\
%$\pi$ 					& \\
%$\varrho$ 				& \\
%$\varsigma$ 			& \\
%$\upsilon$ 			& \\
$\Upsilon_\varepsilon$ 	& Simplifying parameter for the upper bound $C_\varepsilon$\\
$\upsilon$ 				& Order of Bessel function\\
$\phi$ 					& First potential function in solid\\
$\vec{\psi}$ 			& Second potential function in solid\\
$\Psi$, $\vec{\Psi}$ 	& General scalar or vector function\\
$\omega_{\mathrm{c}}$ 	& Center angular frequency in wavelet\\
$\varphi$ 				& Azimuth angle in the spherical coordinate system\\
$\varphi_{\mathrm{s}}$ 	& Azimuth (or aspect) angle for the source point\\
\end{tabularx}

\section*{Notation for paper II}
\begin{tabularx}{\linewidth}{p{0.12\linewidth} X}
$d_1$, $d_2$ 			& Distances from foci in prolate coordinate system\\
$R_0$ 					& Outer surface radius of spherical shell\\
$R_1$ 					& Inner surface radius of spherical shell\\
$w$ 					& Weights for weighted Sobolev spaces in $\Omega^+$ for the trial space\\
$w^*$ 					& Weights for weighted Sobolev spaces in $\Omega^+$ for the test space\\
$p_1$ 					& Scattered pressure in exterior domain $\Omega^+$\\
$p_2$ 					& Scattered pressure in interior domain $\Omega^-$\\
$Q_m$ 					& Polynomial function of the radial shape functions $\phi$\\
$\tilde{Q}_n$ 			& Polynomial function of the radial shape functions $\psi$\\
$\Gamma_{\mathrm{a}}$ 	& Artificial boundary\\
$\Gamma_0$				& Outer boundary of scatterer\\
$\Gamma_1$				& Inner boundary of scatterer\\
$\phi_n$ 				& Basis function in radial direction in infinite elements for the trial space\\
$\psi_n$ 				& Basis function in radial direction in infinite elements for the test space\\
$\Omega_{\mathrm{a}} $ 	& Fluid domain inside artificial boundary $\Gamma_{\mathrm{a}}$\\
$\Omega_{\mathrm{a}}^+$ & Fluid domain outside the artificial boundary $\Gamma_{\mathrm{a}}$\\
\end{tabularx}

\section*{Notation for paper III}
\begin{tabularx}{\linewidth}{p{0.12\linewidth} X}
$C^\pm$ 			& Jump-term in the exterior ($+$) or interior ($-$) problem\\
$f_t$ 				& NACA profile function of foil with width $t$\\
$l$ 				& Distance from the center of an element to a source point\\
$n_{\mathrm{eqp},1}$& Number of extra quadrature points in BEM formulation (in addition to the standard $\check{p}+1$ points) in elements not containing singularity\\
$n_{\mathrm{eqp},2}$& Number of extra quadrature points in BEM formulation (in addition to the standard $\check{p}+1$ points) in elements containing singularity\\
$N$ 				& Number of terms in manufactured solution\\
$s_1$				& Parameter controlling adaptivity of quadrature integration\\
$\gamma^+$ 			& Trace operator in $\Omega^+$\\
$\gamma^-$ 			& Trace operator in $\Omega^-$\\
$\alpha$ 			& Coupling parameter in BM formulation\\
$\Psi$ 				& Jump-term in the exterior (+) or interior (-) problem\\
\end{tabularx}

\section*{Notation for paper IV}
\begin{tabularx}{\linewidth}{p{0.12\linewidth} X}
$f(\cdot)$ 			& Amplitude function in oscillatory integrals\\
$F(\cdot)$ 			& Non-oscillatory exponentially integral\\
$g(\cdot)$ 			& Oscillator function in oscillatory integrals\\
$h_\xi$ 			& Path in the complex plane at which the oscillatory integral is not oscillatory\\
$\alpha$ 			& Exponent of weight factor in GGL quadrature\\
\end{tabularx}

\section*{Notation for addendum}
\begin{tabularx}{\linewidth}{p{0.12\linewidth} X}
$f(\cdot)$ 				& Right hand side function for the Poisson equation\\
$l_{i,\vec{t}}$ 		& The $i^{\mathrm{th}}$ Lagrange basis function defined over the interpolation points in $\vec{t}$\\
$\Phi_n$ 				& Generating polynomial\\
\end{tabularx}

\section*{Notation for appendices}
\begin{tabularx}{\linewidth}{p{0.12\linewidth} X}
$a$			 			& Longitudinal angular wave number for the first potential function (in the solid)\\
$b$						& Transverse angular wave number for the second potential function (in the solid)\\
$E_j$ 					& Energy of beam $B_j$\\
$h$ 					& Solid layer thickness\\
$N$ 					& Number of rays/beams\\
$R$ 					& Reflection coefficient\\
$s$ 					& Parameter for parameterizing rays\\
$T$ 					& Transmission coefficient\\
$\theta_1$ 				& Angle of incidence\\
$\theta_2$ 				& Angle of refraction\\
$\theta_3$ 				& Angle of transmission\\
$\tau$ 					& Phase function in ray series\\
$\phi$ 					& First potential function in solid\\
$\psi$ 					& Second potential function in solid\\
\end{tabularx}
% Greek alphapet: Α α, Β β, Γ γ, Δ δ, Ε ε, Ζ ζ, Η η, Θ θ, Ι ι, Κ κ, Λ λ, Μ μ, Ν ν, Ξ ξ, Ο ο, Π π, Ρ ρ, Σ σ/ς, Τ τ, Υ υ, Φ φ, Χ χ, Ψ ψ, and Ω ω