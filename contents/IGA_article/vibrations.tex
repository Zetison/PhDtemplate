\section{Isogeometric analysis of vibrations}
\label{Sec2:vibrations}
A fundamental mathematical assumption of this thesis, is the harmonic time dependency assumption (see \Cref{Eq2:periodicityAssumption}). When we transform our time dependent problem to a frequency dependent problem, it is advantageous to analyze the phenomena of vibrations (also named \textit{spectrum analysis}). As it turns out, there are examples where IGA is demonstrably better in this analysis compared to ordinary FEA. That is, one may wonder if the NURBS basis functions (with their smooth properties) serve a better basis than the typical Lagrange basis function which are commonly used in FEM. We shall illustrate this improvement by an example taken from~\cite{Cottrell2009iat}. The work in~\cite{hughes2009imi} and~\cite{Cottrell2006iao} investigates the improvement of IGA even further.

\subsection{Longitudinal vibrations of an elastic rod}
\label{subse:longVibElRod}
We shall consider an one dimensional rod of length $L$ such that our spatial domain is given by $\Omega = (0, L)$, where we apply homogeneous Dirichlet boundary conditions. That is, $u=0$ at $\Gamma = \partial\Omega = \{0, L\}$. In one dimension \Cref{Eq2:mainDiffEqn} reads
\begin{equation}\label{Eq2:elastic1Drod}
	E\pderiv[2]{u}{x} + \omega^2 \rho u = 0
\end{equation}
where it is assumed that $E$ and $\rho$ are constant throughout the domain. Proceeding in the usual way, we construct a solution space $\mathcal{S}=\{u|u\in H^1(\Omega), u|_\Gamma = 0\}$. As we have homogeneous Dirichlet boundary conditions, the weighted space $\mathcal{V}$ is identical to the solution space $\mathcal{S}$. We now multiply \Cref{Eq2:elastic1Drod} by an arbitrary $v\in \mathcal{V}$ and integrate the resulting equation over the domain.
\begin{equation*}
	\int_0^L  vE\pderiv[2]{u}{x} \idiff x + \omega^2\int_0^L v\rho u \idiff x = 0
\end{equation*}
Using partial integration we get
\begin{equation*}
	\left[vE\pderiv{u}{x} \right]_0^L - \int_0^L \pderiv{v}{x}E \pderiv{u}{x} \idiff x+ \omega^2\int_0^L v \rho u  \idiff x = 0
\end{equation*}
Exploiting now the fact that $v(0) = v(L) = 0$, the resulting weak form is given by
\begin{equation*}
	a(v,u) -\omega^2 (v,\rho u) = 0
\end{equation*}
where
\begin{align*}
	a(v,u) &= \int_0^L  \pderiv{v}{x} E\pderiv{u}{x} \idiff x\\
	( v,\rho u) &= \omega^2\int_0^L  v\rho u \idiff x
\end{align*}
Turning now to the Galerkin formulation, we restrict ourselves to solution of the form
\begin{equation*}
	u^h = \sum_{A = 1}^{n_{\mathrm{eq}}} N_A d_A\quad\text{and}\quad u^h = \sum_{B = 1}^{n_{\mathrm{eq}}} N_B d_B
\end{equation*}
such that our eigenvalue problem now becomes: Find $\omega^h\in\R^+$ and $u^h\in\mathcal{S}^h$ such that for all $v^h\in\mathcal{V}^h$ we have
\begin{equation*}
	a(v^h,u^h) -(\omega^h)^2 (u^h,\rho v^h) = 0
\end{equation*}
In the analysis we use a nonlinear parametrization of the rod such that the control points are uniformly placed in the physical space. As noted in~\cite{Cottrell2009iat}, this result in better results when studying structural vibrations. 

Using a finite number of NURBS basis functions\footnote{As all the weights are 1, we actually only get simple B-splines; a subset of NURBS functions.} ($n_{\mathrm{eq}}$), we will get the same number of eigenvalues $\omega_n^h$.

\pgfplotsset{
	legend style={
		at={(0.03,0.97)},
		anchor=north west,
		legend columns=1,
		%cells={anchor=west},
		font=\footnotesize,
		rounded corners=2pt
	}
}
\tikzsetnextfilename{longVibRodFEMIGAcomparison}
\begin{figure}
	\centering
	\begin{tikzpicture}
		\begin{axis}[
			xtick={0, 0.1, 0.2, 0.3, 0.4, 0.5, 0.6, 0.7, 0.8, 0.9, 1},
			xticklabels={0, 0.1, 0.2, 0.3, 0.4, 0.5, 0.6, 0.7, 0.8, 0.9, 1},
			%ytick={0.001,0.01, 0.1, 1.0, 10.0, 100.0},
			%yticklabels={0.001,0.01, 0.1, 1.0, 10.0, 100.0},
			cycle list={%
				{myYellow}, 
				{mycolor}, 
				{mycolor, dashed}, 
				{myGreen}, 
				{myGreen,dashed}, 
				{myCyan},
				{myCyan,dashed},
				{myRed},
				{myRed,dashed}
			},
			width = 0.8\textwidth,
			height = 0.3\paperheight,
			%xticklabels={0.2, 0.15, 0.1, 0.05, 0},
			%xmin=80, 
			% ymin=0.0007,
			% xmax=200000, 
			xlabel=$n/n_{\mathrm{eq}}$,
			ylabel=$\omega_n^h/\omega_n$]
			
			\addplot table[x=n,y=omega,mark=none] {../../data/plotData/vibrationRodIGA/p1.dat};
			\addlegendentry{$p = 1$}
			
			\addplot table[x=n,y=omega,mark=none] {../../data/plotData/vibrationRodIGA/p2.dat};
			\addlegendentry{IGA: $p = 2$}
			\addplot table[x=n,y=omega,mark=none] {../../data/plotData/vibrationRodFEM/p2.dat};
			\addlegendentry{FEM: $p = 2$}
			
			\addplot table[x=n,y=omega,mark=none] {../../data/plotData/vibrationRodIGA/p3.dat};
			\addlegendentry{IGA: $p = 3$}
			\addplot table[x=n,y=omega,mark=none] {../../data/plotData/vibrationRodFEM/p3.dat};
			\addlegendentry{FEM: $p = 3$}
			
			\addplot table[x=n,y=omega,mark=none] {../../data/plotData/vibrationRodIGA/p4.dat};
			\addlegendentry{IGA: $p = 4$}
			\addplot table[x=n,y=omega,mark=none] {../../data/plotData/vibrationRodFEM/p4.dat};
			\addlegendentry{FEM: $p = 4$}
			
			\addplot table[x=n,y=omega,mark=none] {../../data/plotData/vibrationRodIGA/p5.dat};
			\addlegendentry{IGA: $p = 5$}
			\addplot table[x=n,y=omega,mark=none] {../../data/plotData/vibrationRodFEM/p5.dat};
			\addlegendentry{FEM: $p = 5$}
		\end{axis}
	\end{tikzpicture}
	\caption[IGA/FEM comparison of longitudinal vibrations of an elastic rod]{\textbf{Longitudinal vibrations of an elastic rod:} Comparison of FEM and IGA. Note that the basis functions for both FEM and IGA are equal in the case $p=1$. The results are thus identical in this case.}\label{Fig2:longVibRodFEMIGAcomparison}
\end{figure}

The solution of the \textit{eigenvalue problem} in \Cref{Eq2:elastic1Drod} is trigonometric equations. With the boundary conditions we get the eigenfunctions $u_n(x) = \sin \left(\sqrt{\frac{\rho}{E}}\omega_n x\right)$ and eigenvalues $\omega_n = \sqrt{\frac{E}{\rho}}\frac{n\pi}{L}$. We consider the case $L=1$, $\rho=1$, $E=1$ and $n_{\mathrm{eq}} = 999$. Then the eigenvalues are simply $\omega_n = n\pi$. We want to compare these analytic eigenvalues to our numerical eigenvalues $\omega_n^h$, by looking at the ratio $\omega_n/\omega_n^h$, which should tend to 1 as the numerical solution becomes better. In \Cref{Fig2:longVibRodFEMIGAcomparison} we plot this ratio against the ratio $n/n_{\mathrm{eq}}$ using standard Lagrange basis functions (FEM) and B-splines (IGA). A motivation for using IGA in stead of FEM in this thesis is quite apparent in this plot. It should be noted that the B-spline basis functions have larger support and the resulting matrix would then be less sparse. This will in turn increase the computational time for finding the eigenvalues. So the smooth results of IGA comes at some cost in computational time of solving the resulting system of equations. For a more detailed explanation of the different behaviors in \Cref{Fig2:longVibRodFEMIGAcomparison}, we refer to Cotterell et al.~\cite{Cottrell2009iat}. 


\subsection{Elastic vibration in 3D}
Moving on to three dimensions, we multiply each of the equations in \eqref{Eq:mainDiffEqn2} by a corresponding test functions $v_i\in\mathcal{S}_i$ and sum these three equations into one single equation given by
\begin{equation*}
	v_i\sigma_{ij,j} +\omega^2\rho_{\mathrm{s}} v_i u_i = 0.
\end{equation*}
Note that we have set $f_i=0$ as we shall not consider any body forces. 

Integration over the domain and using the same procedure as for the static linear elasticity case, get the linear form
\begin{equation*}
	L(\vec{v}) = \int_{\Gamma_N} v_i h_j\idiff\Gamma.
\end{equation*}
and the bilinear form
\begin{equation*}
	a(\vec{v},\vec{u}) = \int_\Omega v_{(i,j)}c_{ijkl}u_{(k,l)}\idiff\Omega - \omega^2\rho_{\mathrm{s}} \int_\Omega v_i u_i\idiff\Omega
\end{equation*}
where the only difference from before is the last term which will result in the \textit{mass matrix} $\vec{M}$. So if $\vec{K}$ is the same stiffness matrix as in the static linear elasticity case, we now have the following eigenvalue problem: Find natural frequency $\omega_k^h\in\R^+$ and eigenvectors $\vec{U}_k$ such that
\begin{equation*}
	(\vec{K}-[\omega_k^h]^2\vec{M})\vec{U}_k=\zerovec
\end{equation*}
where (using the same notation as before) the mass matrix is given by
\begin{equation*}
	\vec{M} = \left[M_{PQ}\right]
\end{equation*}
and
\begin{equation*}
	M_{PQ} = \rho_{\mathrm{s}} \vec{e}_i\int_\Omega R_A R_B\idiff\Omega\vec{e}_j.
\end{equation*}
As $\vec{e}_i\cdot \vec{e}_j = \delta_{ij}$ we may write
\begin{equation*}
	M_{PQ} = \rho_{\mathrm{s}} \delta_{ij}\int_\Omega R_A R_B\idiff\Omega.
\end{equation*}
In the same way the stiffness element matrix could be computed in one go, so can the \textit{element mass matrix}:
\begin{equation*}
	\vec{m}^e = \rho_{\mathrm{s}} \int_{\tilde{\Omega}^e}\mathrm{blkdiag}\left(\vec{R}^\transpose\vec{R},\vec{R}^\transpose\vec{R},\vec{R}^\transpose\vec{R}\right)|J_1||J_2| \idiff\tilde{\Omega}.
\end{equation*}
where (once again)
\begin{equation*}
	\vec{R}^\transpose = \begin{bmatrix}
		R_1, & R_2, & \dots, & R_{n_{\mathrm{en}}}
	\end{bmatrix}
\end{equation*}
and $\mathrm{blkdiag}()$ creates a block diagonal matrix with it's arguments on the diagonal.

\subsubsection{Circular plate vibrating in vacuum}
\begin{table}
	\centering
	\caption[Parameters for circular plate]{Parameters for circular plate.}
	\label{Tab2:circularPlateParameters}
	\begin{tabular}{l l}
		\toprule
		Parameter & Description\\
		\midrule
		$E = 30\cdot 10^9\unit{Pa}$ & Young's modulus\\
		$\rho = 2.32\cdot 10^3\unit{kg/m^3}$ & Density\\
		$\nu = 0.2$ & Poisson's ratio\\
		$t = 0.02\unit{m}$ & Thickness of the plate\\
		$R = 2\unit{m}$ & Radius of plate\\
		\bottomrule
	\end{tabular}
\end{table}
In~\cite[pp. 185-189]{Meirovitch1967ami} we find the exact Poisson-Kirchhoff solution to the clamped circular plate. One finds the eigenvalues by first solving
\begin{equation*}
	\imag^{-n}J_n(\imag x) J_{n-1}(\imag x) - \imag^{-n-1}J_n(x) J_{n-1}(\imag x) = 0,\quad n = 0,1,2,\dots
\end{equation*}
where $J_n$ are the mentioned Bessel functions of the first kind. This equation has infinitely many solution (as for the spherical shell), so we obtain a set of solutions $\{x_{nm}\}$. The eigenvalues are then given by $\beta_{nm} = x_{nm}/R$ such that the natural frequencies may be found by
\begin{equation}
	\omega_{nm} = \beta_{nm}^2\sqrt{\frac{D_E}{\rho t}},\quad D_E = \frac{E t^3}{12(1-\nu^2)}.
\end{equation}
In~\cite{Hughes2005iac} the notation $C_{nm} = x_{nm}/\pi$ has been used. In~\cite[p. 188]{Meirovitch1967ami} there is a misprint in the first eigenfrequency: $C_{01} = 1.015$ where the true value is $C_{01} = 1.0173886$. Hughes et al. copies this misprint in~\cite{Hughes2005iac} such that he compares his result with $\omega_{01} = 53.863$ when the true value is $\omega_{01} = 54.117$. Hence, in~\cite{Hughes2005iac} there seems to be convergence to the wrong result when this is not the case.
\begin{table}
	\centering
	\caption[Eigenvalues for circular plate]{\textbf{Circular plate:} Eigenvalues.}
	\label{Tab2:circularPlateFirstEigenValues}
	\pgfplotstableset{% global config, for example in the preamble
	% these columns/<colname>/.style={<options>} things define a style
	% which applies to <colname> only.
	columns/n/.style={int detect,column type=r,column name=\textsc{$n$}}, 
	columns/omega_n_0/.style={/pgf/number format/fixed, sci sep align, column name=\textsc{$\omega_{n1}$}},
	columns/omega_n_1/.style={/pgf/number format/fixed, sci sep align, column name=\textsc{$\omega_{n2}$}},
	columns/omega_n_2/.style={/pgf/number format/fixed, sci sep align, column name=\textsc{$\omega_{n3}$}},
	columns/omega_n_3/.style={/pgf/number format/fixed, sci sep align, column name=\textsc{$\omega_{n4}$}},
	every head row/.style={before row=\toprule,after row=\midrule},
	every last row/.style={after row=\bottomrule}
	}
	\pgfplotstabletypeset[fixed zerofill,precision=6, 1000 sep={\,},1000 sep in fractionals]{../../data/plotData/circularPlate/sortedEigenvalues.dat}
\end{table}
The NURBS data for the coarsest geometry of the plate shell may be found in \Cref{se:circularPlateNURBSdata}, here two parametrizations for the circular plate is presented (cf. \Cref{Fig2:circularPlate1_new} and \Cref{Fig2:circularPlate2_new}). In~\cite{Schmidt2010roa} better convergence results was shown for parametrization 2, but we shall restrict ourselves to the analysis on the first of these parametrizations as the main objective is to compare with the results in~~\cite{Cottrell2009iat}. Some deviation in the comparison of the results occured, but the main trend is the same.

In \Cref{Tab2:circlarPlateComputedEigenfrequencies} we show the $p$-convergence of the first 4 computed frequencies. The corresponding modes are shown in \Cref{Fig2:EigenModesCircularPlate}. We can here observe that for axisymmetric modes ($\omega_{0n}$, $n=0,1,2,\dots$) there is hardly any accuracy to be gained by increasing the polynomial order in the angular direction ($\xi$-direction).
\begin{table}
	\centering
	\caption[Convergence of computed eigenfrequencies for circular plate]{\textbf{Circular plate:} Convergence of computed eigenfrequencies.}
	\label{Tab2:circlarPlateComputedEigenfrequencies}
	\begin{tabular}{S S S S S S S}
		\toprule
		    {$p$} & {$q$} & {$r$} & {$\omega_{01}$} & {$\omega_{11}$} & {$\omega_{21}$} & {$\omega_{02}$}\\
		\midrule
						2 & 2 & 2 & 254.838 & 778.860 & 2569.078 & 3456.036\\	
						2 & 3 & 2 & 54.424 & 138.174 & 400.934 & 214.661\\	
						3 & 3 & 2 & 54.424 & 117.358 & 196.698 & 214.661\\	
						3 & 4 & 2 & 54.253 & 116.359 & 187.731 & 211.796\\	
						4 & 4 & 2 & 54.253 & 112.954 & 185.972 & 211.796\\	
						4 & 5 & 2 & 54.203 & 112.801 & 185.475 & 210.912\\	
						5 & 5 & 2 & 54.203 & 112.783 & 185.218 & 210.912\\	
						5 & 6 & 2 & 54.177 & 112.728 & 185.122 & 210.803\\	
		\multicolumn{3}{c}{exact} & 54.117 & 112.624 & 184.756 & 210.682\\	
	   %\multicolumn{3}{c}{exact} & 53.863 & 112.670 & {\textcolor{red}{??}} & 210.597\\	
		\bottomrule
	\end{tabular}
\end{table}
\begin{figure}
	\centering    
	\begin{subfigure}{0.47\textwidth}
		\centering   
		\includegraphics[width=0.8\textwidth]{../../graphics/circularPlate_2}
		\caption{Parametrization 1.}
		\label{Fig2:circularPlate1_new}
	\end{subfigure}
	~ 
	\begin{subfigure}{0.47\textwidth}
		\centering   
		\includegraphics[width=0.8\textwidth]{../../graphics/circularPlate2_2}
		\caption{Parametrization 2.}
		\label{Fig2:circularPlate2_new}
	\end{subfigure}
	\caption[Parametrizations of circular plate]{\textbf{Circular plate}: Example of parametrizations of the circular plate.}
\end{figure}

\begin{figure}
	\centering        
	\begin{subfigure}{0.47\textwidth}
		\centering   
		\includegraphics[width=0.7\textwidth]{../../graphics/vibrationCircularPlateMode1}
		\caption{Mode 1.}
	\end{subfigure}
	~
	\begin{subfigure}{0.47\textwidth}
		\centering   
		\includegraphics[width=0.7\textwidth]{../../graphics/vibrationCircularPlateMode2}
		\caption{Mode 2.}
	\end{subfigure}
	\par\bigskip
	\begin{subfigure}{0.47\textwidth}
		\centering   
		\includegraphics[width=0.7\textwidth]{../../graphics/vibrationCircularPlateMode4}
		\caption{Mode 4.}
	\end{subfigure}
	~ 
	\begin{subfigure}{0.47\textwidth}
		\centering   
		\includegraphics[width=0.7\textwidth]{../../graphics/vibrationCircularPlateMode6}
		\caption{Mode 6.}
	\end{subfigure}
	\caption[Cibration modes of circular plate]{\textbf{Circular plate}: Four vibration modes using $p=6$, $q = 6$ and $r = 2$ with parametrization 1.}
	\label{Fig2:EigenModesCircularPlate}
\end{figure}

\subsubsection{Spherical shell vibrating in vacuum}
We shall repeat the analysis we did on the circular plate with the spherical shell with data from \Cref{Tab2:sphericalShellParameters}. The NURBS data for the coarsest geometry of the spherical shell may be found in \Cref{se:sphericalShellNURBSData}. In this analysis we shall insert the knots $\{\frac{1}{8}, \frac{3}{8}, \frac{5}{8}, \frac{7}{8}\}$ in the $\xi$-direction and $\{0.25, 0.75\}$ in the $\eta$-direction. We do not refine the mesh in $\zeta$-direction. This results in mesh 2 in \Cref{Fig2:shericalShellMesh2}. 

The results is shown in \Cref{Tab2:sphericalShellComputedEigenfrequencies} where the exact eigenfrequencies are collected from \Cref{Tab2:sphericalShellFirstEigenValues}. As the spherical shell is not clamped, the first 6 modes (3 from translation and 3 from rotation) will correspond to eigenvalue equal zero. The first interesting mode is thus vibration mode 7, which corresponds to $\omega_{21}$. Up to rotation, there is 4 identical other modes (identical eigenvalues), such that the next mode is vibration mode 12 which then corresponds to $\omega_{31}$. The first four unique modes up to rotation is shown in \Cref{Fig2:EigenModesSphericalShell}. We note that the accuracy decreases for modes with higher frequencies.
\begin{table}
	\centering
	\caption[Convergence of computed eigenfrequencies for spherical shell]{\textbf{Spherical shell:} Convergence of computed eigenfrequencies.}
	\label{Tab2:sphericalShellComputedEigenfrequencies}
	\begin{tabular}{S S S S S S S}
		\toprule
		{$p$} & {$q$} & {$r$} & {$\omega_{21}$} & {$\omega_{31}$} & {$\omega_{41}$} & {$\omega_{51}$}\\
		\midrule
		2 & 2 & 2 & 777.802 & 1019.834 & 1379.161 & 2159.979\\	
		3 & 3 & 2 & 764.428 & 913.290 & 1000.350 & 1235.107\\	
		4 & 4 & 2 & 764.175 & 908.693 & 976.975 & 1038.282\\	
		5 & 5 & 2 & 764.123 & 908.041 & 972.766 & 1021.684\\	
		\multicolumn{3}{c}{exact} & 764.121 & 907.967 & 972.169 & 1017.829\\	
		\bottomrule
	\end{tabular}
\end{table}
\begin{figure}
	\centering        
	\begin{subfigure}{0.47\textwidth}
		\centering   
		\includegraphics[width=0.7\textwidth]{../../graphics/sphericalShellMesh1}
		\caption{Mesh 1.}
		\label{Fig2:shericalShellMesh2_1}
	\end{subfigure}
	~ 
	\begin{subfigure}{0.47\textwidth}
		\centering   
		\includegraphics[width=0.7\textwidth]{../../graphics/sphericalShellMesh2}
		\caption{Mesh 2.}
	\end{subfigure}
	\caption[Meshes of a spherical shell]{\textbf{Spherical shell:} The first two meshes with, respectively, 8 and 32 elements.}
	\label{Fig2:shericalShellMesh2}
	\par\bigskip	
	\begin{subfigure}{0.47\textwidth}
		\centering   
		\includegraphics[width=0.7\textwidth]{../../graphics/vibrSphericalShell7}
		\caption{Mode 7 corresponding to $\omega_{21}$.}
	\end{subfigure}
	~
	\begin{subfigure}{0.47\textwidth}
		\centering   
		\includegraphics[width=0.7\textwidth]{../../graphics/vibrSphericalShell12}
		\caption{Mode 12 corresponding to $\omega_{31}$.}
	\end{subfigure}
	\par\bigskip
	\begin{subfigure}{0.47\textwidth}
		\centering   
		\includegraphics[width=0.7\textwidth]{../../graphics/vibrSphericalShell19}
		\caption{Mode 19 corresponding to $\omega_{41}$.}
	\end{subfigure}
	~
	\begin{subfigure}{0.47\textwidth}
		\centering   
		\includegraphics[width=0.7\textwidth]{../../graphics/vibrSphericalShell28}
		\caption{Mode 28 corresponding to $\omega_{51}$.}
	\end{subfigure}
	\caption[Vibration modes of spherical shell]{\textbf{Spherical shell}: First four unique (up to rotation) vibration modes using $p=5$, $q=5$ and $r=2$. The colors corresponds to displacement in the radial direction.}
	\label{Fig2:EigenModesSphericalShell}
\end{figure}