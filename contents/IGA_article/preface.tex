\section{Preface}
This thesis concludes my master's degree in Mathematics at the Norwegian University of Technology and Science (NTNU). The work spans both the fall 2014 and the spring 2015. The thesis represents the second half of the 2 year master program, which is a continuation of my bachelor's in degree at NTNU.

The thesis was proposed by my supervisor Trond Jenserud\footnote{Researcher at Norwegian Defence Research Establishment (FFI).}. At that time, I had experience both with FEM and IGA as I had been following three courses involving these methods. These courses would set the foundation of the knowledge needed to complete a thesis on the subject. All of these courses was supervised by Trond Kvamsdal\footnote{Professor in Computational Mathematics at Department of Mathematical Sciences, NTNU.}, and it was thus natural to choose Kvamsdal as an adviser at NTNU. Although two complete programs had been made for FEM and IGA (both involving solution of elasticity problems) in these courses, I decided to start from scratch by building each part of the code in a structural way. This way, I was able to create generic functions which could solve a larger set of problems. To test the validity of the implementations, many test examples was executed, mostly from the article and book of Hughes et al. (\cite{Hughes2005iac} and~\cite{Cottrell2009iat}). Some additional test examples were invented to validate special aspects of the implementations. These include linear elasticity problem on a Solid circular cylinder and rectangular prisms. This groundwork then enabled more complicated simulations of the vibration analysis, exterior Helmholtz problems and finally fully structure interaction problems. 

The work of this thesis has been to consider the application of IGA. After all, the chosen specialization in the master's degree is applied mathematics. One typically presents the foundation of the theory from functional analysis where the relevant spaces are a topic of discussion. However, I shall in these cases simply refer to other literature.

This thesis forms a basis for a huge study that could be made on the subject. Not only are there topics to be investigated in regards to IGA on acoustic scattering, but also the coupling of IGA and absorbing boundary conditions. In this thesis I choose the infinite element method, but there are many other candidates which could work very well with IGA. Much of this study will be a topic of my PhD which will start this fall.

I would like to express my appreciation to the BeTSSi community for providing a good description for the models to be analyzed, alongside a well arranged setup for the results obtained. I would also like express my appreciation for the two books in~\cite{Cottrell2009iat} and~\cite{Ihlenburg1998fea} by Cotterell et al. and Frank Ihlenburg, respectively. The first has been my main reference to IGA, while the second has been my main reference for acoustic scattering using FEA.

I would like to thank NTNU for a perfect course of study, and in particular I want to thank the Department of Mathematical Sciences for providing an environment with hardware capable of computationally expensive simulations, which have been essential for this thesis. I would also thank Ilkka Karasalo and Martin Østberg\footnote{Researchers at Swedish Defence Research Agency (FOI).}  for providing validating data sets for the final simulations and enthusiastic support. Moreover, I would like to express my gratitude to my two supervisors Trond Jenserud and Trond Kvamsdal for their encouragement and guidance. Finally, I would like to thank Kjetil Andre Johannessen\footnote{Postdoctoral at NTNU.} for his inspirational encouragement which has not only been essential for my master thesis, but also ignited my interest in FEM/IGA in the early stages.
\vspace{2cm}
\begin{flushright}
Jon Vegard Venås\\
NTNU, Trondheim\\
\today
\end{flushright}
