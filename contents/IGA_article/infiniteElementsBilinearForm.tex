\newpage
\section{Derivation of bilinear form in infinite elements}
\label{Sec2:AppendixDerivationOfBilinearForm}
In this appendix, the integrals in the bilinear forms for the infinite elements will be separated for the PGU case\footnote{The other three formulations have been derived in~\cite{Venas2015iao}. For the more general ellipsoidal coordinate system, refer to~\cite{Burnett1998aea}.}. For generality, the derivation is done in the prolate spheroidal coordinate system.

\subsection{The prolate spheroidal coordinate system}
\label{Sec2:prolateSphericalCoordinateSystem}
The prolate spheroidal coordinate system is an extension of the spherical coordinate system. It is defined by the relations
\begin{align}
	x &= \sqrt{r^2 - \Upsilon^2}\sin\vartheta\cos\varphi\\
	y &= \sqrt{r^2 - \Upsilon^2}\sin\vartheta\sin\varphi\\
	z &= r\cos\vartheta
\end{align}
with foci located at $z = \pm \Upsilon$ and $r\geq \Upsilon$. Note that the coordinate system reduces to the spherical coordinate system when $\Upsilon=0$. the following inverse formulas may be derived
\begin{align}\label{Eq2:XtoProl}
\begin{split}
	r &= \frac{1}{2}(d_1+d_2)\\
	\vartheta &= \arccos\left(\frac{z}{r}\right)\\
	\varphi &= \operatorname{atan2}(y,x)
\end{split}
\end{align}
where 
\begin{align*}
	d_1 &= d_1(x,y,z) = \sqrt{x^2+y^2+(z+\Upsilon)^2}\\
	d_2 &= d_2(x,y,z) = \sqrt{x^2+y^2+(z-\Upsilon)^2}
\end{align*}
and 
\begin{equation*}
	\operatorname{atan2}(y,x) = \begin{cases}
	\arctan(\frac{y}{x}) & \mbox{if } x > 0\\
	\arctan(\frac{y}{x}) + \PI & \mbox{if } x < 0 \mbox{ and } y \ge 0\\
	\arctan(\frac{y}{x}) - \PI & \mbox{if } x < 0 \mbox{ and } y < 0\\
	\frac{\PI}{2} & \mbox{if } x = 0 \mbox{ and } y > 0\\
	-\frac{\PI}{2} & \mbox{if } x = 0 \mbox{ and } y < 0\\
	\text{undefined} & \mbox{if } x = 0 \mbox{ and } y = 0.
	\end{cases}
\end{equation*}
The derivatives are found to be
\begin{equation}\label{Eq2:dXdProlateSphericalCoordinates}
\begin{alignedat}{4}
	\pderiv{x}{r} &= \frac{r\sin\vartheta\cos\varphi}{\sqrt{r^2-\Upsilon^2}},\qquad	&&\pderiv{y}{r} = \frac{r\sin\vartheta\sin\varphi}{\sqrt{r^2-\Upsilon^2}},\qquad	&&\pderiv{z}{r} = \cos\vartheta\\
	\pderiv{x}{\vartheta} &=\sqrt{r^2-\Upsilon^2}\cos\vartheta\cos\varphi,\qquad	&&\pderiv{y}{\vartheta} = \sqrt{r^2-\Upsilon^2}\cos\vartheta\sin\varphi,\qquad	&&\pderiv{z}{\vartheta} = -r\sin\vartheta\\
	\pderiv{x}{\varphi} &= -\sqrt{r^2-\Upsilon^2}\sin\vartheta\sin\varphi,\qquad	&&\pderiv{y}{\varphi} = \sqrt{r^2-\Upsilon^2}\sin\vartheta\cos\varphi,\qquad	&&\pderiv{z}{\varphi} =0
\end{alignedat}
\end{equation}
and
\begin{equation}\label{Eq2:dProlateSphericalCoordinatesdX}
\begin{aligned}
	\pderiv{r}{x} &= \frac{x(d_1+d_2)}{2d_1d_2},\qquad	\pderiv{r}{y} = \frac{y(d_1+d_2)}{2d_1d_2}\qquad\\
	\pderiv{r}{z} &= \frac{z(d_1+d_2)+\Upsilon(d_2-d_1)}{2d_1d_2}\\
	\pderiv{\vartheta}{x} &= \frac{xz}{d_1d_2\sqrt{r^2-z^2}},\qquad	\pderiv{\vartheta}{y} = \frac{yz}{d_1d_2\sqrt{r^2-z^2}}	\\
	\pderiv{\vartheta}{z} &= \frac{1}{\sqrt{r^2-z^2}}\left(\frac{z^2}{d_1d_2}+\frac{\Upsilon z(d_2-d_1)}{d_1d_2(d_1+d_2)}-1\right) \\
	\pderiv{\varphi}{x} &= -\frac{y}{x^2+y^2},\qquad	\pderiv{\varphi}{y} = \frac{x}{x^2+y^2},\qquad	\pderiv{\varphi}{z} = 0.
\end{aligned}
\end{equation}
The general nabla operator can be written as
\begin{equation}\label{Eq2:GeneralNablaOperator}
	\nabla = \frac{\vec{e}_{\mathrm{r}}}{h_{\mathrm{r}}} \pderiv{}{r} + \frac{\vec{e}_{\upvartheta}}{h_{\upvartheta}} \pderiv{}{\vartheta} + \frac{\vec{e}_{\upvarphi}}{h_{\upvarphi}} \pderiv{}{\varphi}
\end{equation}
where
\begin{equation*}
	\vec{e}_{\mathrm{r}} = \frac{1}{h_{\mathrm{r}}}\left[\pderiv{x}{r}, \pderiv{y}{r}, \pderiv{z}{r}\right]^\transpose,\quad
	\vec{e}_{\upvartheta} = \frac{1}{h_{\upvartheta}}\left[\pderiv{x}{\vartheta}, \pderiv{y}{\vartheta}, \pderiv{z}{\vartheta}\right]^\transpose,\quad
	\vec{e}_{\upvarphi} = \frac{1}{h_{\upvarphi}}\left[\pderiv{x}{\varphi}, \pderiv{y}{\varphi}, \pderiv{z}{\varphi}\right]^\transpose
\end{equation*}
and
\begin{align*}
	h_{\mathrm{r}} &= \sqrt{\frac{r^2-\Upsilon^2\cos^2\vartheta}{r^2-\Upsilon^2}}\\
	h_{\upvartheta} &= \sqrt{r^2-\Upsilon^2\cos^2\vartheta}\\
	h_{\upvarphi} &= \sqrt{r^2-\Upsilon^2}\sin\vartheta.
\end{align*}
The Jacobian determinant (for the mapping from Cartesian coordinates to prolate spheroidal coordinates) may now be written as
\begin{equation}
	J_1 = h_{\mathrm{r}} h_{\upvartheta} h_{\upvarphi} = \left(r^2-\Upsilon^2\cos^2\vartheta\right)\sin\vartheta.
\end{equation}
As any normal vector at a surface with constant radius $r=\gamma$ can be written as $\vec{n} = \vec{e}_{\upvartheta}\times\vec{e}_{\upphi}=\vec{e}_{\mathrm{r}}$
\begin{equation}
	\partial_n p = \vec{n}\cdot\nabla p = \vec{e}_{\mathrm{r}}\cdot\nabla p = \frac{1}{h_{\mathrm{r}}} \pderiv{p}{r}.
\end{equation}
The surface Jacobian determinant at a given (constant) $r=\gamma$ is
\begin{equation}
	J_S = h_{\upvartheta} h_{\upvarphi} = \sqrt{r^2-\Upsilon^2\cos^2\vartheta}\sqrt{r^2-\Upsilon^2}\sin\vartheta,
\end{equation}
such that
\begin{equation}
	q\partial_n p J_S = \bigoh\left(r^{-3}\right)\quad\text{whenever}\quad q=\bigoh\left(r^{-3}\right)\quad\text{and}\quad p = \bigoh\left(r^{-1}\right).
\end{equation}
That is, for the Petrov--Galerkin formulations
\begin{equation}
	\lim_{\gamma\to\infty}\int_{S^\gamma} q\partial_n p\idiff\Gamma = \lim_{\gamma\to\infty}\int_0^{2\PI}\int_0^\PI q\partial_n p J_S\idiff\vartheta\idiff\varphi = 0.
\end{equation}

\subsection{Bilinear form for unconjugated Petrov--Galerkin formulation}
The bilinear form (in the domain outside the artificial boundary) in~\Cref{Eq2:B_uc_a} (in the unconjugated case) can in the Petrov--Galerkin formulations be simplified to
\begin{align}\label{Eq2:BilinearFormInserted}
	\begin{split}
	B_{\textsc{pgu}}(R_I\psi_n,R_J\phi_m) &= \lim_{\gamma\to\infty}\int_{\Omega_{\mathrm{a}}^\gamma} \left[\nabla(R_I\psi_n)\cdot \nabla (R_J\phi_m)- k^2 R_I\psi_n R_J\phi_m\right]\idiff\Omega\\
		&=\int_{\Omega_{\mathrm{a}}^+} \left[\nabla(R_I\psi_n)\cdot \nabla (R_J\phi_m)- k^2 R_I\psi_n R_J\phi_m\right]\idiff\Omega
	\end{split}
\end{align}
as the mentioned surface integral in the far field vanishes (this is however not the case for the Bubnov--Galerkin formulations). Recall that the radial shape functions are given by
\begin{align*}
	\phi_m(r) &= \euler^{\imag k (r-r_{\mathrm{a}})}Q_m\left(\frac{r_{\mathrm{a}}}{r}\right),\quad m = 1,\dots,N\\
	\psi_n(r) &= \euler^{\imag k (r-r_{\mathrm{a}})}\tilde{Q}_n\left(\frac{r_{\mathrm{a}}}{r}\right),\quad n = 1,\dots,N
\end{align*}
such that the derivative can be computed by
\begin{equation*}
	\deriv{\phi_m}{r} = \left[\imag kQ_m\left(\frac{r_{\mathrm{a}}}{r}\right) - \frac{r_{\mathrm{a}}}{r^2}Q_m'\left(\frac{r_{\mathrm{a}}}{r}\right)\right]\euler^{\imag k (r-r_{\mathrm{a}})}
\end{equation*}
and corresponding expression for $\psi_n$. Using the expression for the nabla operator found in \Cref{Eq2:GeneralNablaOperator}
\begin{align*}
	\nabla(R_I\psi_n)\cdot \nabla (R_J\phi_m) &= \frac{1}{h_{\mathrm{r}}^2}\pderiv{(R_I\psi_n)}{r}\pderiv{(R_J\phi_m)}{r} + \frac{1}{h_{\uptheta}^2}\pderiv{(R_I\psi_n)}{\vartheta}\pderiv{(R_J\phi_m)}{\vartheta} \\
	&{\hskip6em\relax}+ \frac{1}{h_{\upvarphi}^2}\pderiv{(R_I\psi_n)}{\varphi}\pderiv{(R_J\phi_m)}{\varphi}\\
	 &= \frac{1}{h_{\mathrm{r}}^2}\pderiv{\psi_n}{r}\pderiv{\phi_m}{r}R_IR_J + \frac{1}{h_{\uptheta}^2}\psi_n\phi_m\pderiv{R_I}{\vartheta}\pderiv{R_J}{\vartheta}\\
	 &{\hskip6em\relax}+ \frac{1}{h_{\upvarphi}^2}\psi_n\phi_m\pderiv{R_I}{\varphi}\pderiv{R_J}{\varphi}
\end{align*}
which multiplied with the Jacobian $J_1$ yields
\begin{align*}
	\nabla(R_I\psi_n)\cdot \nabla (R_J\phi_m) J_1&= \left[\left(r^2-\Upsilon^2\right)\pderiv{\psi_n}{r}\pderiv{\phi_m}{r}R_IR_J + \psi_n\phi_m\pderiv{R_I}{\vartheta}\pderiv{R_J}{\vartheta}\right. \\
	 &{\hskip8em\relax}\left.+ \frac{r^2-\Upsilon^2\cos^2\vartheta}{(r^2-\Upsilon^2)\sin^2\vartheta}\psi_n\phi_m\pderiv{R_I}{\varphi}\pderiv{R_J}{\varphi}\right]\sin\vartheta
\end{align*}
Combining all of this into \Cref{Eq2:BilinearFormInserted} yields
\begin{align}
	B_{\textsc{pgu}}(R_I\psi_n,R_J\phi_m) = &\int_0^{2\PI}\int_0^\PI K(\vartheta,\varphi)\sin\vartheta\idiff\vartheta\idiff\varphi
\end{align}
where
\begin{align*}
	K(\vartheta,\varphi) &= \int_{r_{\mathrm{a}}}^{\infty} \left\{\left(r^2-\Upsilon^2\right)\pderiv{\psi_n}{r}\pderiv{\phi_m}{r}R_IR_J + \psi_n\phi_m\pderiv{R_I}{\vartheta}\pderiv{R_J}{\vartheta}\right. \\
	 &{\hskip4em\relax}\left.+ \frac{r^2-\Upsilon^2\cos^2\vartheta}{(r^2-\Upsilon^2)\sin^2\vartheta}\psi_n\phi_m\pderiv{R_I}{\varphi}\pderiv{R_J}{\varphi}\right.\\
	 &{\hskip4em\relax}\left.-k^2(r^2-\Upsilon^2\cos^2\vartheta)\psi_n\phi_m R_I R_J\right\}\idiff r.
\end{align*}
Inserting the expressions for the radial shape functions $\phi$ and $\psi$ (with Einstein's summation convention) with their corresponding derivatives one obtains the following expression using the substitution $\rho = \frac{r}{r_{\mathrm{a}}}$ and the notation $\varrho_1 = \Upsilon/r_{\mathrm{a}}$ (the eccentricity of the infinite-element spheroid), $\varrho_2=kr_{\mathrm{a}}$ and $\varrho_3=k\Upsilon$
\begin{align*} 
	 K(\vartheta,\varphi) &= \left\{R_IR_J\left[-2\varrho_2^2B_{\tilde{n}+\tilde{m}}^{(1)} - \imag \varrho_2(\tilde{n}+\tilde{m}+2)B_{\tilde{n}+\tilde{m}+1}^{(1)} \phantom{\left(\pderiv{r_{\mathrm{a}}}{\varphi}\right)^2} \right.\right.\\
	 &\left.\left.{\hskip5em\relax} + \left[\tilde{m}(\tilde{n}+2) +\varrho_3^2\right]B_{\tilde{n}+\tilde{m}+2}^{(1)}+ \imag \varrho_1^2\varrho_2(\tilde{n}+\tilde{m}+2)B_{\tilde{n}+\tilde{m}+3}^{(1)}\right.\right.\\
	 &\left.\left.{\hskip5em\relax}  - \tilde{m}(\tilde{n}+2)\varrho_1^2 B_{\tilde{n}+\tilde{m}+4}^{(1)}+\varrho_3^2\cos^2\vartheta B_{\tilde{n}+\tilde{m}+2}^{(1)}\right] \right.\\
	 &\left.{\hskip2em\relax}+ \pderiv{R_I}{\vartheta}\pderiv{R_J}{\vartheta}B_{\tilde{n}+\tilde{m}+2}^{(1)}\right.\\
	 &\left.{\hskip2em\relax}+\pderiv{R_I}{\varphi}\pderiv{R_J}{\varphi}\frac{1}{\sin^2\vartheta}\left(B_{\tilde{n}+\tilde{m}+1}^{(2)} - \varrho_1^2\cos^2\vartheta B_{\tilde{n}+\tilde{m}+3}^{(2)}\right)\right\}r_{\mathrm{a}}\euler^{-2\imag \varrho_2}\tilde{D}_{n\tilde{n}}D_{m\tilde{m}}
\end{align*}
where the radial integrals
\begin{equation*}
	B_{n}^{(1)} = \int_{1}^\infty \frac{\euler^{2\imag \varrho_2 \rho}}{\rho^n}\idiff \rho \qquad B_{n}^{(2)} = \int_{1}^\infty \frac{\euler^{2\imag \varrho_2\rho}}{(\rho^2-\varrho_1^2)\rho^{n-1}}\idiff \rho,\qquad n\geq 1
\end{equation*}
can be evaluated according to formulas in \Cref{Sec2:radIntegrals}.

Assume that the artificial boundary $\Gamma_{\mathrm{a}}$ is parameterized by $\xi$ and $\eta$. As $\Gamma_{\mathrm{a}}$ is a surface with constant radius, $r=r_{\mathrm{a}}$, in the prolate spheroidal coordinate system, it may also be parameterized by $\vartheta$ and $\varphi$. Therefore,
\begin{equation}
	\diff \vartheta\diff\varphi = \begin{vmatrix}
		\pderiv{\vartheta}{\xi} & \pderiv{\varphi}{\xi}\\
		\pderiv{\vartheta}{\eta} & \pderiv{\varphi}{\eta}		
	\end{vmatrix}\diff\xi\diff\eta
\end{equation}
where
\begin{align*}
	\pderiv{\vartheta}{\xi} &= \pderiv{\vartheta}{x}\pderiv{x}{\xi} + \pderiv{\vartheta}{y}\pderiv{y}{\xi} + \pderiv{\vartheta}{z}\pderiv{z}{\xi},\qquad
	\pderiv{\vartheta}{\eta} = \pderiv{\vartheta}{x}\pderiv{x}{\eta} + \pderiv{\vartheta}{y}\pderiv{y}{\eta} + \pderiv{\vartheta}{z}\pderiv{z}{\eta}\\
	\pderiv{\varphi}{\xi} &= \pderiv{\varphi}{x}\pderiv{x}{\xi} + \pderiv{\varphi}{y}\pderiv{y}{\xi} + \pderiv{\varphi}{z}\pderiv{z}{\xi},\qquad
	\pderiv{\varphi}{\eta} = \pderiv{\varphi}{x}\pderiv{x}{\eta} + \pderiv{\varphi}{y}\pderiv{y}{\eta} + \pderiv{\varphi}{z}\pderiv{z}{\eta}
\end{align*}
and the inverse partial derivatives with respect to the coordinate transformation (from the prolate spheroidal coordinate system to the Cartesian coordinate system) is found in \Cref{Eq2:dProlateSphericalCoordinatesdX}. This Jacobian matrix may be evaluated by
\begin{equation}
	J_3 = \begin{bmatrix}
		\pderiv{\vartheta}{\xi} & \pderiv{\vartheta}{\eta}\\
		\pderiv{\varphi}{\xi}	 & \pderiv{\varphi}{\eta}
	\end{bmatrix} = \begin{bmatrix}
		\pderiv{\vartheta}{x} & \pderiv{\vartheta}{y} & \pderiv{\vartheta}{z}\\
		\pderiv{\varphi}{x} & \pderiv{\varphi}{y} & \pderiv{\varphi}{z}
	\end{bmatrix}\begin{bmatrix}
		\pderiv{x}{\xi} & \pderiv{x}{\eta}\\
		\pderiv{y}{\xi} & \pderiv{y}{\eta}\\
		\pderiv{z}{\xi} & \pderiv{z}{\eta}
	\end{bmatrix}
\end{equation}
and the derivatives of the basis functions may then be computed by
\begin{equation}
	\begin{bmatrix}
		\pderiv{R_I}{\vartheta}\\
		\pderiv{R_I}{\varphi}
	\end{bmatrix} = J_3^{-\transpose}\begin{bmatrix}
		\pderiv{R_I}{\xi}\\
		\pderiv{R_I}{\eta}	
	\end{bmatrix}.
\end{equation}
Defining the angular integrals
\begin{equation}\label{Eq2:infiniteElementsSurfaceIntegrals}
\begin{alignedat}{2}
	& A_{IJ}^{(1)} = \int_0^{2\PI}\int_0^\PI R_I R_J\sin\vartheta\idiff\vartheta\idiff\varphi,\qquad\qquad && A_{IJ}^{(2)} = \int_0^{2\PI}\int_0^\PI \pderiv{R_I}{\vartheta} \pderiv{R_J}{\vartheta}\sin\vartheta\idiff\vartheta\idiff\varphi\\
	& A_{IJ}^{(3)} = \int_0^{2\PI}\int_0^\PI R_I R_J\cos^2\vartheta\sin\vartheta\idiff\vartheta\idiff\varphi, \quad && A_{IJ}^{(4)} = \int_0^{2\PI}\int_0^\PI \pderiv{R_I}{\varphi} \pderiv{R_J}{\varphi}\frac{1}{\sin\vartheta}\idiff\vartheta\idiff\varphi\\
	& A_{IJ}^{(5)} = \int_0^{2\PI}\int_0^\PI \pderiv{R_I}{\varphi} \pderiv{R_J}{\varphi}\frac{\cos^2\vartheta}{\sin\vartheta}\idiff\vartheta\idiff\varphi
\end{alignedat}	
\end{equation}
the bilinear form may then finally be written as (Einstein's summation convention is used for the indices $\tilde{n}$ and $\tilde{m}$)
\begin{align}\label{Eq2:finalBilinearFormB_uc_a}
\begin{split}
	&B_{\textsc{pgu}}(R_I\psi_n,R_J\phi_m) \\
	 &= \left\{A_{IJ}^{(1)}\left[-2\varrho_2^2 B_{\tilde{n}+\tilde{m}}^{(1)} - \imag\varrho_2(\tilde{n}+\tilde{m}+2)B_{\tilde{n}+\tilde{m}+1}^{(1)} + \left[(\tilde{n}+2)\tilde{m}+\varrho_3^2\right]B_{\tilde{n}+\tilde{m}+2}^{(1)}\right. \right. \\
	  &{\hskip4em\relax}\left. \left. + \imag\varrho_1\varrho_3(\tilde{n}+\tilde{m}+2)B_{\tilde{n}+\tilde{m}+3}^{(1)} - \varrho_1^2(\tilde{n}+2)\tilde{m}B_{\tilde{n}+\tilde{m}+4}^{(1)}\right] \right. \\
	 &{\hskip2em\relax}\left. +A_{IJ}^{(2)}B_{\tilde{n}+\tilde{m}+2}^{(1)} + \varrho_3^2A_{IJ}^{(3)} B_{\tilde{n}+\tilde{m}+2}^{(1)}\right.\\
	 &{\hskip2em\relax}\left. +A_{IJ}^{(4)}B_{\tilde{n}+\tilde{m}+1}^{(2)} -\varrho_1^2A_{IJ}^{(5)} B_{\tilde{n}+\tilde{m}+3}^{(2)}
	 \right\}r_{\mathrm{a}}\euler^{-2\imag \varrho_2}D_{m\tilde{m}}\tilde{D}_{n\tilde{n}}.
\end{split}
\end{align}
For completeness, the formulas for the other three formulations are included
\begin{equation}
\begin{aligned}
	&B_{\textsc{bgu}}(R_I\psi_n,R_J\phi_m) \\
	 &= \left\{A_{IJ}^{(1)}\left[-2\varrho_2^2 B_{\tilde{n}+\tilde{m}-2}^{(1)}(1-\delta_{\tilde{n}1}\delta_{\tilde{m}1}) - \imag\varrho_2(\tilde{n}+\tilde{m})B_{\tilde{n}+\tilde{m}-1}^{(1)} + \left(\tilde{n}\tilde{m}+\varrho_3^2\right)B_{\tilde{n}+\tilde{m}}^{(1)}\right. \right. \\
	  &{\hskip4em\relax}\left. \left. + \imag\varrho_1\varrho_3(\tilde{n}+\tilde{m})B_{\tilde{n}+\tilde{m}+1}^{(1)} - \varrho_1^2\tilde{n}\tilde{m}B_{\tilde{n}+\tilde{m}+2}^{(1)}\right] \right. \\
	 &{\hskip2em\relax}\left. +A_{IJ}^{(2)}B_{\tilde{n}+\tilde{m}}^{(1)} + \varrho_3^2A_{IJ}^{(3)} B_{\tilde{n}+\tilde{m}}^{(1)}\right.\\
	 &{\hskip2em\relax}\left. +A_{IJ}^{(4)}B_{\tilde{n}+\tilde{m}-1}^{(2)} -\varrho_1^2A_{IJ}^{(5)} B_{\tilde{n}+\tilde{m}+1}^{(2)}
	 \right\}r_{\mathrm{a}}\euler^{-2\imag \varrho_2}D_{m\tilde{m}}\tilde{D}_{n\tilde{n}}\\
	 &\quad-\imag\varrho_2 r_{\mathrm{a}} D_{m1}\tilde{D}_{n1}A_{IJ}^{(1)}
\end{aligned}
\end{equation}
\begin{equation}
\begin{aligned}
	&B_{\textsc{pgc}}(R_I\psi_n,R_J\phi_m) \\
	 &= \left\{A_{IJ}^{(1)}\left[-\imag\varrho_2(\tilde{n}-\tilde{m}+2)B_{\tilde{n}+\tilde{m}+1}^{(1)} + \left[(\tilde{n}+2)\tilde{m}-\varrho_3^2\right]B_{\tilde{n}+\tilde{m}+2}^{(1)}\right. \right. \\
	  &{\hskip4em\relax}\left. \left. + \imag\varrho_1\varrho_3(\tilde{n}-\tilde{m}+2)B_{\tilde{n}+\tilde{m}+3}^{(1)} - \varrho_1^2(\tilde{n}+2)\tilde{m}B_{\tilde{n}+\tilde{m}+4}^{(1)}\right] \right. \\
	 &{\hskip2em\relax}\left. +A_{IJ}^{(2)}B_{\tilde{n}+\tilde{m}+2}^{(1)} + \varrho_3^2A_{IJ}^{(3)} B_{\tilde{n}+\tilde{m}+2}^{(1)}\right.\\
	 &{\hskip2em\relax}\left. +A_{IJ}^{(4)}B_{\tilde{n}+\tilde{m}+1}^{(2)} -\varrho_1^2A_{IJ}^{(5)} B_{\tilde{n}+\tilde{m}+3}^{(2)}
	 \right\}r_{\mathrm{a}}D_{m\tilde{m}}\tilde{D}_{n\tilde{n}}
\end{aligned}
\end{equation}
\begin{equation}
\begin{aligned}
	B_{\textsc{bgc}}(R_I\psi_n,R_J\phi_m) 
	 &= \left\{A_{IJ}^{(1)}\left[-\imag\varrho_2(\tilde{n}-\tilde{m})B_{\tilde{n}+\tilde{m}-1}^{(1)} + \left(\tilde{n}\tilde{m}-\varrho_3^2\right)B_{\tilde{n}+\tilde{m}}^{(1)}\right. \right. \\
	  &{\hskip4em\relax}\left. \left. + \imag\varrho_1\varrho_3(\tilde{n}-\tilde{m})B_{\tilde{n}+\tilde{m}+1}^{(1)} - \varrho_1^2\tilde{n}\tilde{m}B_{\tilde{n}+\tilde{m}+2}^{(1)}\right] \right. \\
	 &{\hskip2em\relax}\left. +A_{IJ}^{(2)}B_{\tilde{n}+\tilde{m}}^{(1)} + \varrho_3^2A_{IJ}^{(3)} B_{\tilde{n}+\tilde{m}}^{(1)}\right.\\
	 &{\hskip2em\relax}\left. +A_{IJ}^{(4)}B_{\tilde{n}+\tilde{m}-1}^{(2)} -\varrho_1^2A_{IJ}^{(5)} B_{\tilde{n}+\tilde{m}+1}^{(2)}
	 \right\}r_{\mathrm{a}}D_{m\tilde{m}}\tilde{D}_{n\tilde{n}}\\
	 &\quad-\imag r_{\mathrm{a}}\varrho_2 D_{m1}\tilde{D}_{n1}A_{IJ}^{(1)} 
\end{aligned}
\end{equation}
where $\delta_{ij}$ is the Kronecker delta function in \Cref{Eq2:Kronecker}.