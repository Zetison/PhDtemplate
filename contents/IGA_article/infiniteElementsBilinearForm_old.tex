\section{Derivation of bilinear form in infinite elements}
In this appendix we will separate the integrals in the bilinear forms for the infinite elements\footnote{As we mainly use the PGU formulation in this work, we restrict ourselves to the derivation of this formulation. The other three formulations has been derived in~\cite{Venas2015iao}. For the more general ellipsoidal coordinate system, we refer to~\cite{Burnett1998aea}.}. For generality, we do this derivation in the prolate spheroidal coordinate system.
\subsection{The prolate spheroidal coordinate system}
\label{Sec2:prolateSphericalCoordinateSystem}
The prolate spheroidal coordinate system is an extension of the spherical coordinate system. It is defined by the relations
\begin{align*}
	x &= \sqrt{r^2 - \Upsilon^2}\sin\vartheta\cos\varphi\\
	y &= \sqrt{r^2 - \Upsilon^2}\sin\vartheta\sin\varphi\\
	z &= r\cos\vartheta
\end{align*}
with foci located at $z = \pm \Upsilon$. Note that we define $r\geq \Upsilon$ such that the coordinate system reduces to the spherical coordinate system when $\Upsilon=0$. Using this condition, we may establish the following inverse formulas
\begin{align}\label{Eq2:XtoProl}
\begin{split}
	r &= \frac{1}{2}\left(\sqrt{T_\Upsilon-2z\Upsilon}+\sqrt{T_\Upsilon+2z\Upsilon}\right)\\
	\vartheta &= \arccos\left(\frac{z}{r}\right)\\
	\varphi &= \operatorname{atan2}(y,x)
\end{split}
\end{align}
where $T_\Upsilon = T_\Upsilon(x,y,z) = x^2+y^2+z^2+\Upsilon^2$ and 
\begin{equation*}
	\operatorname{atan2}(y,x) = \begin{cases}
	\arctan(\frac{y}{x}) & \mbox{if } x > 0\\
	\arctan(\frac{y}{x}) + \PI & \mbox{if } x < 0 \mbox{ and } y \ge 0\\
	\arctan(\frac{y}{x}) - \PI & \mbox{if } x < 0 \mbox{ and } y < 0\\
	\frac{\PI}{2} & \mbox{if } x = 0 \mbox{ and } y > 0\\
	-\frac{\PI}{2} & \mbox{if } x = 0 \mbox{ and } y < 0\\
	\text{undefined} & \mbox{if } x = 0 \mbox{ and } y = 0.
	\end{cases}
\end{equation*}
The derivatives are found to be
\begin{equation}\label{Eq2:dXdProlateSphericalCoordinates}
\begin{alignedat}{4}
	\pderiv{x}{r} &= \frac{r\sin\vartheta\cos\varphi}{\sqrt{r^2-\Upsilon^2}},\qquad	&&\pderiv{y}{r} = \frac{r\sin\vartheta\sin\varphi}{\sqrt{r^2-\Upsilon^2}},\qquad	&&\pderiv{z}{r} = \cos\vartheta\\
	\pderiv{x}{\vartheta} &=\sqrt{r^2-\Upsilon^2}\cos\vartheta\cos\varphi,\qquad	&&\pderiv{y}{\vartheta} = \sqrt{r^2-\Upsilon^2}\cos\vartheta\sin\varphi,\qquad	&&\pderiv{z}{\vartheta} = -r\sin\vartheta\\
	\pderiv{x}{\varphi} &= -\sqrt{r^2-\Upsilon^2}\sin\vartheta\sin\varphi,\qquad	&&\pderiv{y}{\varphi} = \sqrt{r^2-\Upsilon^2}\sin\vartheta\cos\varphi,\qquad	&&\pderiv{z}{\varphi} =0
\end{alignedat}
\end{equation}
and
\begin{equation}\label{Eq2:dProlateSphericalCoordinatesdX}
\begin{alignedat}{4}
	\pderiv{r}{x} &= \frac{xr}{2r^2-T_\Upsilon},\qquad	&&\pderiv{r}{y} = \frac{yr}{2r^2-T_\Upsilon},\qquad	&&\pderiv{r}{z} = \frac{z}{r}\frac{r^2-\Upsilon^2}{2r^2-T_\Upsilon}\\
	\pderiv{\vartheta}{x} &= \frac{xz}{r\sin\vartheta (2r^2-T_\Upsilon)},\qquad	&&\pderiv{\vartheta}{y} = \frac{yz}{r\sin\vartheta (2r^2-T_\Upsilon)},\qquad	&&\pderiv{\vartheta}{z} = \frac{1}{r\sin\vartheta}\left(\frac{z^2}{r^2}\frac{r^2-\Upsilon^2}{2r^2-T_\Upsilon}-1\right)\\
	\pderiv{\varphi}{x} &= -\frac{y}{x^2+y^2},\qquad	&&\pderiv{\varphi}{y} = \frac{x}{x^2+y^2},\qquad	&&\pderiv{\varphi}{z} = 0.
\end{alignedat}
\end{equation}
With the notation
\begin{align*}
	h_{\mathrm{r}} &= \sqrt{\frac{r^2-\Upsilon^2\cos^2\vartheta}{r^2-\Upsilon^2}}\\
	h_{\upvartheta} &= \sqrt{r^2-\Upsilon^2\cos^2\vartheta}\\
	h_{\upvarphi} &= \sqrt{r^2-\Upsilon^2}\sin\vartheta	
\end{align*}
the general nabla operator can be written as
\begin{equation*}
	\nabla = \frac{\vec{e}_{\mathrm{r}}}{h_{\mathrm{r}}} \pderiv{}{r} + \frac{\vec{e}_{\upvartheta}}{h_{\upvartheta}} \pderiv{}{\vartheta} + \frac{\vec{e}_{\upvarphi}}{h_{\upvarphi}} \pderiv{}{\varphi}
\end{equation*}
where
\begin{equation*}
	\vec{e}_{\mathrm{r}} = \frac{1}{h_{\mathrm{r}}}\left[\pderiv{x}{r}, \pderiv{y}{r}, \pderiv{z}{r}\right]^\transpose,\qquad
	\vec{e}_{\upvartheta} = \frac{1}{h_{\upvartheta}}\left[\pderiv{x}{\vartheta}, \pderiv{y}{\vartheta}, \pderiv{z}{\vartheta}\right]^\transpose,\qquad
	\vec{e}_{\upvarphi} = \frac{1}{h_{\upvarphi}}\left[\pderiv{x}{\varphi}, \pderiv{y}{\varphi}, \pderiv{z}{\varphi}\right]^\transpose.
\end{equation*}
The Jacobian determinant (for the mapping from Cartesian coordinates to prolate spheroidal coordinates) may now be written as
\begin{equation*}
	J_1 = h_{\mathrm{r}} h_{\upvartheta} h_{\upvarphi} = \left(r^2-\Upsilon^2\cos^2\vartheta\right)\sin\vartheta.
\end{equation*}
By noting that any normal vector at a surface with constant radius $r=\gamma$ can be written as $\vec{n} = \vec{e}_{\upvartheta}\times\vec{e}_{\upphi}=\vec{e}_{\mathrm{r}}$ we get
\begin{equation*}
	\partial_n p = \vec{n}\cdot\nabla p = \vec{e}_{\mathrm{r}}\cdot\nabla p = \frac{1}{h_{\mathrm{r}}} \pderiv{p}{r}.
\end{equation*}
The surface Jacobian determinant at a given (constant) $r=\gamma$ is
\begin{equation*}
	J_S = h_{\upvartheta} h_{\upvarphi} = \sqrt{r^2-\Upsilon^2\cos^2\vartheta}\sqrt{r^2-\Upsilon^2}\sin\vartheta	.
\end{equation*}
and we may therefore note that
\begin{equation*}
	q\partial_n p J_S = \bigoh\left(r^{-3}\right)\quad\text{whenever}\quad q=\bigoh\left(r^{-3}\right)\quad\text{and}\quad p = \bigoh\left(r^{-1}\right).
\end{equation*}
That is, for the Petrov--Galerkin formulations we have
\begin{equation*}
	\lim_{\gamma\to\infty}\int_{S^\gamma} q\partial_n p\idiff\Gamma = \lim_{\gamma\to\infty}\int_0^{2\PI}\int_0^\PI q\partial_n p J_S\idiff\vartheta\idiff\varphi = 0.
\end{equation*}

\subsection{Bilinear form}
The bilinear form (in the domain outside the artificial boundary) in~\Cref{Eq2:B_uc_a} (in the unconjugated case) can in the Petrov--Galerkin formulations be simplified to
\begin{align}\label{Eq2:BilinearFormInserted}
	\begin{split}
	B_{\mathrm{uc},\mathrm{a}}(R_I\psi_n,R_J\phi_m) &= \lim_{\gamma\to\infty}\int_{\Omega_{\mathrm{a}}^\gamma} \left[\nabla(R_I\psi_n)\cdot \nabla (R_J\phi_m)- k^2 R_I\psi_n R_J\phi_m\right]\idiff\Omega\\
		&=\int_{\Omega_{\mathrm{a}}^+} \left[\nabla(R_I\psi_n)\cdot \nabla (R_J\phi_m)- k^2 R_I\psi_n R_J\phi_m\right]\idiff\Omega
	\end{split}
\end{align}
as the mentioned surface integral in the far-field vanishes (this is however not the case for the Bubnov--Galerkin formulations). With the summation convention (over the indices $\tilde{n}$ and $\tilde{m}$) we can write
\begin{equation*}
	\psi_n = \euler^{\imag k(r-r_n)}\frac{\tilde{D}_{n\tilde{n}}}{r^{\tilde{n}+2}}\quad\text{and}\quad\phi_m = \euler^{\imag k(r-r_m)}\frac{D_{m\tilde{m}}}{r^{\tilde{m}}}.
\end{equation*}
Using the expression for the nabla operator found in \Cref{Sec2:prolateSphericalCoordinateSystem} we get
\begin{align*}
	\nabla(R_I\psi_n)\cdot \nabla (R_J\phi_m) &= \frac{1}{h_{\mathrm{r}}^2}\pderiv{(R_I\psi_n)}{r}\pderiv{(R_J\phi_m)}{r} + \frac{1}{h_{\upvartheta}^2}\pderiv{(R_I\psi_n)}{\vartheta}\pderiv{(R_J\phi_m)}{\vartheta} + \frac{1}{h_{\upvarphi}^2}\pderiv{(R_I\psi_n)}{\varphi}\pderiv{(R_J\phi_m)}{\varphi}\\
	 &= \frac{1}{h_{\mathrm{r}}^2}\pderiv{\psi_n}{r}\pderiv{\phi_m}{r}R_IR_J + \frac{1}{h_{\upvartheta}^2}\pderiv{R_I}{\vartheta}\pderiv{R_J}{\vartheta}\phi_m\psi_n + \frac{1}{h_{\upvarphi}^2}\pderiv{R_I}{\varphi}\pderiv{R_J}{\varphi}\psi_n\phi_m\\
	 &=  \left(\imag k - \frac{\tilde{n}+2}{r}\right)\left(\imag k - \frac{\tilde{m}}{r}\right)\frac{\euler^{2\imag k r}\euler^{-\imag k (r_n+r_m)}\tilde{D}_{n\tilde{n}}D_{m\tilde{m}}}{r^{\tilde{n}+\tilde{m}+2}h_{\mathrm{r}}^2}R_I R_J \\
	 &\quad+ \frac{\euler^{2\imag k r}\euler^{-\imag k (r_n+r_m)}\tilde{D}_{n\tilde{n}}D_{m\tilde{m}}}{h_{\upvartheta}^2 r^{\tilde{n}+\tilde{m}+2}}\pderiv{R_I}{\vartheta}\pderiv{R_J}{\vartheta} + \frac{\euler^{2\imag k r}\euler^{-\imag k (r_n+r_m)}\tilde{D}_{n\tilde{n}}D_{m\tilde{m}}}{h_{\upvarphi}^2 r^{\tilde{n}+\tilde{m}+2}}\pderiv{R_I}{\varphi}\pderiv{R_J}{\varphi}\\
	 &= \left[\frac{(r^2-\Upsilon^2)\left[-(kr)^2 - \imag kr(\tilde{n}+\tilde{m}+2) + (\tilde{n}+2)\tilde{m}\right]}{r^2(r^2-\Upsilon^2\cos^2\vartheta)}\frac{\euler^{2\imag k r}}{r^{\tilde{n}+\tilde{m}+2}}R_I R_J\right.\\ 
	 &\qquad\left.+ \frac{1}{r^2-\Upsilon^2\cos^2\vartheta}\frac{\euler^{2\imag k r}}{r^{\tilde{n}+\tilde{m}+2}}\pderiv{R_I}{\vartheta}\pderiv{R_J}{\vartheta} \right.\\
	 &\qquad\left.+ \frac{1}{(r^2-\Upsilon^2)\sin^2\vartheta}\frac{\euler^{2\imag k r}}{r^{\tilde{n}+\tilde{m}+2}}\pderiv{R_I}{\varphi}\pderiv{R_J}{\varphi}\right]\tilde{D}_{n\tilde{n}}D_{m\tilde{m}}\euler^{-\imag k (r_n+r_m)}
\end{align*}
which multiplied with the Jacobian $J_1$ yields
\begin{align*}
	\nabla(R_I\psi_n)\cdot \nabla (R_J\phi_m) J_1&= \left[\frac{(r^2-\Upsilon^2)\left[-(kr)^2 - \imag kr(\tilde{n}+\tilde{m}+2) + (\tilde{n}+2)\tilde{m}\right]}{r^2}\frac{\euler^{2\imag k r}}{r^{\tilde{n}+\tilde{m}+2}}\sin\vartheta R_I R_J\right.\\ 
	&\qquad+ \left.\frac{\euler^{2\imag k r}}{r^{\tilde{n}+\tilde{m}+2}}\sin\vartheta \pderiv{R_I}{\vartheta}\pderiv{R_J}{\vartheta} \right.\\
	&\qquad+\left. \frac{(r^2-\Upsilon^2\cos^2\vartheta)}{(r^2-\Upsilon^2)\sin\vartheta}\frac{\euler^{2\imag k r}}{r^{\tilde{n}+\tilde{m}+2}}\pderiv{R_I}{\varphi}\pderiv{R_J}{\varphi}\right]\tilde{D}_{n\tilde{n}}D_{m\tilde{m}}\euler^{-\imag k (r_n+r_m)}
\end{align*}
Also note that the term contributing to the mass matrix multiplied with the same Jacobian is given by
\begin{equation*}
	k^2R_I\psi_n R_J\phi_m J_1 = \frac{k^2\euler^{2\imag k r}}{r^{\tilde{n}+\tilde{m}+2}} (r^2-\Upsilon^2\cos^2\vartheta)\sin\vartheta R_I R_J\tilde{D}_{n\tilde{n}}D_{m\tilde{m}}\euler^{-\imag k (r_n+r_m)}.
\end{equation*}
Combining all of this into \Cref{Eq2:BilinearFormInserted} yields
\begin{align*}
	B_{\mathrm{uc},\mathrm{a}}(R_I\psi_n,R_J\phi_m) =&\left\{\int_{r_{\mathrm{a}}}^{\infty}\left[-\frac{k^2}{r^{\tilde{n}+\tilde{m}}} - \frac{\imag k(\tilde{n}+\tilde{m}+2)}{r^{\tilde{n}+\tilde{m}+1}} +\frac{(\tilde{n}+2)\tilde{m}+\Upsilon^2k^2}{r^{\tilde{n}+\tilde{m}+2}}\right.\right.\\
	&\left.\left.\quad\qquad + \frac{\imag k \Upsilon^2 (\tilde{n}+\tilde{m}+2)}{r^{\tilde{n}+\tilde{m}+3}} - \frac{(\tilde{n}+2)\tilde{m}\Upsilon^2}{r^{\tilde{n}+\tilde{m}+4}}\right]\euler^{2\imag k r}\idiff r \int_0^{2\PI}\int_0^\PI R_I R_J\sin\vartheta\idiff\vartheta\idiff\varphi\right.\\
	&\left.\quad +\int_{r_{\mathrm{a}}}^{\infty}\frac{\euler^{2\imag k r}}{r^{\tilde{n}+\tilde{m}+2}}\idiff r\int_0^{2\PI}\int_0^\PI \pderiv{R_I}{\vartheta} \pderiv{R_J}{\vartheta}\sin\vartheta\idiff\vartheta\idiff\varphi\right.\\
	&\left.\quad +\int_{r_{\mathrm{a}}}^{\infty}\frac{\euler^{2\imag k r}}{(r^2-\Upsilon^2)r^{\tilde{n}+\tilde{m}}}\idiff r\int_0^{2\PI}\int_0^\PI \pderiv{R_I}{\varphi} \pderiv{R_J}{\varphi}\frac{1}{\sin\vartheta}\idiff\vartheta\idiff\varphi \right.\\
	&\left.\quad- \int_{r_{\mathrm{a}}}^{\infty}\frac{\euler^{2\imag k r}\Upsilon^2}{(r^2-\Upsilon^2)r^{\tilde{n}+\tilde{m}+2}}\idiff r\int_0^{2\PI}\int_0^\PI \pderiv{R_I}{\varphi} \pderiv{R_J}{\varphi}\frac{\cos^2\vartheta}{\sin\vartheta}\idiff\vartheta\idiff\varphi \right.\\
	&\left.\quad- \int_{r_{\mathrm{a}}}^{\infty}\frac{k^2\euler^{2\imag k r}}{r^{\tilde{n}+\tilde{m}}}\idiff r\int_0^{2\PI}\int_0^\PI R_I R_J\sin\vartheta\idiff\vartheta\idiff\varphi\right.\\
	&\left.\quad+ \int_{r_{\mathrm{a}}}^{\infty}\frac{k^2\Upsilon^2\euler^{2\imag k r}}{r^{\tilde{n}+\tilde{m}+2}}\idiff r\int_0^{2\PI}\int_0^\PI R_I R_J\cos^2\vartheta\sin\vartheta\idiff\vartheta\idiff\varphi\right\}\tilde{D}_{n\tilde{n}}D_{m\tilde{m}}\euler^{-\imag k (r_n+r_m)}.
\end{align*}
Defining the angular integrals
\begin{equation}\label{Eq2:infiniteElementsSurfaceIntegrals}
\begin{alignedat}{2}
	& A_{IJ}^{(1)} = \int_0^{2\PI}\int_0^\PI R_I R_J\sin\vartheta\idiff\vartheta\idiff\varphi,\qquad\qquad && A_{IJ}^{(2)} = \int_0^{2\PI}\int_0^\PI \pderiv{R_I}{\vartheta} \pderiv{R_J}{\vartheta}\sin\vartheta\idiff\vartheta\idiff\varphi\\
	& A_{IJ}^{(3)} = \int_0^{2\PI}\int_0^\PI \pderiv{R_I}{\varphi} \pderiv{R_J}{\varphi}\frac{1}{\sin\vartheta}\idiff\vartheta\idiff\varphi, \quad && A_{IJ}^{(4)} = \int_0^{2\PI}\int_0^\PI \pderiv{R_I}{\varphi} \pderiv{R_J}{\varphi}\frac{\cos^2\vartheta}{\sin\vartheta}\idiff\vartheta\idiff\varphi\\
	& A_{IJ}^{(5)} = \int_0^{2\PI}\int_0^\PI R_I R_J\cos^2\vartheta\sin\vartheta\idiff\vartheta\idiff\varphi
\end{alignedat}	
\end{equation}
we can write
\begin{align*}
	B_{\mathrm{uc},\mathrm{a}}(R_I\psi_n,R_J\phi_m) =&\left\{A_{IJ}^{(1)}\int_{r_{\mathrm{a}}}^{\infty}\left[-\frac{k^2}{r^{\tilde{n}+\tilde{m}}} - \frac{\imag k(\tilde{n}+\tilde{m}+2)}{r^{\tilde{n}+\tilde{m}+1}} +\frac{(\tilde{n}+2)\tilde{m}+\Upsilon^2k^2}{r^{\tilde{n}+\tilde{m}+2}}\right.\right.\\
	&\left.\left.\quad\qquad\qquad + \frac{\imag k \Upsilon^2 (\tilde{n}+\tilde{m}+2))}{r^{\tilde{n}+\tilde{m}+3}} - \frac{(\tilde{n}+2)\tilde{m}\Upsilon^2}{r^{\tilde{n}+\tilde{m}+4}}\right]\euler^{2\imag k r}\idiff r\right.\\
	&\left.\quad + A_{IJ}^{(2)}\int_{r_{\mathrm{a}}}^{\infty}\frac{\euler^{2\imag k r}}{r^{\tilde{n}+\tilde{m}+2}}\idiff r +A_{IJ}^{(3)}\int_{r_{\mathrm{a}}}^{\infty}\frac{\euler^{2\imag k r}}{(r^2-\Upsilon^2)r^{\tilde{n}+\tilde{m}}}\idiff r\right.\\
	&\left.\quad - A_{IJ}^{(4)}\int_{r_{\mathrm{a}}}^{\infty}\frac{\euler^{2\imag k r}\Upsilon^2}{(r^2-\Upsilon^2)r^{\tilde{n}+\tilde{m}+2}}\idiff r\right.\\
	&\left.\quad- A_{IJ}^{(1)}\int_{r_{\mathrm{a}}}^{\infty}\frac{k^2\euler^{2\imag k r}}{r^{\tilde{n}+\tilde{m}}}\idiff r + A_{IJ}^{(5)}\int_{r_{\mathrm{a}}}^{\infty}\frac{k^2\Upsilon^2\euler^{2\imag k r}}{r^{\tilde{n}+\tilde{m}+2}}\idiff r\right\}\tilde{D}_{n\tilde{n}}D_{m\tilde{m}}\euler^{-\imag k (r_n+r_m)}.
\end{align*}
Integrals of the form
\begin{equation*}
	\int_{r_{\mathrm{a}}}^\infty\frac{\euler^{2\imag k r}}{r^n}\idiff r,\quad n\geq 1
\end{equation*}
and
\begin{equation}\label{Eq2:2ndIntegralType}
	\int_{r_{\mathrm{a}}}^\infty\frac{\euler^{2\imag k r}}{(r^2-\Upsilon^2)r^n}\idiff r, \quad n\geq 0
\end{equation}
can be evaluated using the algorithms presented in \Cref{Sec2:radIntegrals}. Using the following notation for the radial integrals
\begin{equation*}
	B_{n}^{(1)} = \int_{r_{\mathrm{a}}}^\infty \frac{\euler^{2\imag k r}}{r^n}\idiff r \qquad B_{n}^{(2)} = \int_{r_{\mathrm{a}}}^\infty \frac{\euler^{2\imag k r}}{(r^2-\Upsilon^2)r^{n-1}}\idiff r,\qquad n\geq 1
\end{equation*}
we may now finally write the bilinear form as
\begin{align}\label{Eq2:finalBilinearFormB_uc_a}
\begin{split}
	B_{\mathrm{uc},\mathrm{a}}(R_I\psi_n,R_J\phi_m) = &\sum_{\tilde{n}=1}^N\sum_{\tilde{m}=1}^N\left\{A_{IJ}^{(1)}\left[-2k^2B_{\tilde{n}+\tilde{m}}^{(1)}-\imag k(\tilde{n}+\tilde{m}+2)B_{\tilde{n}+\tilde{m}+1}^{(1)} \right.\right.\\
	&\left.\left.\qquad\qquad\qquad + \left((\tilde{n}+2)\tilde{m}+\Upsilon^2k^2\right)B_{\tilde{n}+\tilde{m}+2}^{(1)} \right.\right.\\
	&\left.\left.\qquad\qquad\qquad + \imag k \Upsilon^2(\tilde{n}+\tilde{m}+2)B_{\tilde{n}+\tilde{m}+3}^{(1)}-(\tilde{n}+2)\tilde{m} \Upsilon^2 B_{\tilde{n}+\tilde{m}+4}^{(1)}\right]\right.\\
	&\left.\qquad\qquad + A_{IJ}^{(2)} B_{\tilde{n}+\tilde{m}+2}^{(1)} +A_{IJ}^{(3)}B_{\tilde{n}+\tilde{m}+1}^{(2)} - A_{IJ}^{(4)} \Upsilon^2B_{\tilde{n}+\tilde{m}+3}^{(2)} \right.\\
	&\left.\qquad\qquad + A_{IJ}^{(5)} k^2 \Upsilon^2 B_{\tilde{n}+\tilde{m}+2}^{(1)}\right\}\tilde{D}_{n\tilde{n}}D_{m\tilde{m}}\euler^{-\imag k (r_n+r_m)}.
\end{split}
\end{align}
We typically parameterize the domain such that the artificial boundary $\Gamma_{\mathrm{a}}$ is parameterized by $\xi$ and $\eta$ at $\zeta=1$ (which is at $r_{\mathrm{a}}$ in the prolate spheroidal coordinates). As $\Gamma_{\mathrm{a}}$ is a surface with constant radius, $r=r_{\mathrm{a}}$, in the prolate spheroidal coordinate system, it may also be parametrized by $\vartheta$ and $\varphi$. We therefore have
\begin{equation}
	\diff \vartheta\diff\varphi = \begin{vmatrix}
		\pderiv{\vartheta}{\xi} & \pderiv{\varphi}{\xi}\\
		\pderiv{\vartheta}{\eta} & \pderiv{\varphi}{\eta}		
	\end{vmatrix}\diff\xi\diff\eta
\end{equation}
where
\begin{align*}
	\pderiv{\vartheta}{\xi} &= \pderiv{\vartheta}{x}\pderiv{x}{\xi} + \pderiv{\vartheta}{y}\pderiv{y}{\xi} + \pderiv{\vartheta}{z}\pderiv{z}{\xi},\qquad
	\pderiv{\vartheta}{\eta} = \pderiv{\vartheta}{x}\pderiv{x}{\eta} + \pderiv{\vartheta}{y}\pderiv{y}{\eta} + \pderiv{\vartheta}{z}\pderiv{z}{\eta}\\
	\pderiv{\varphi}{\xi} &= \pderiv{\varphi}{x}\pderiv{x}{\xi} + \pderiv{\varphi}{y}\pderiv{y}{\xi} + \pderiv{\varphi}{z}\pderiv{z}{\xi},\qquad
	\pderiv{\varphi}{\eta} = \pderiv{\varphi}{x}\pderiv{x}{\eta} + \pderiv{\varphi}{y}\pderiv{y}{\eta} + \pderiv{\varphi}{z}\pderiv{z}{\eta}
\end{align*}
where the inverse partial derivatives with respect to the coordinate transformation\footnote{From the prolate spheroidal coordinate system to the Cartesian coordinate system.\label{Fn:CartToProl}} is found in \Cref{Eq2:dProlateSphericalCoordinatesdX}. This Jacobian matrix may be evaluated by
\begin{equation*}
	J_3 = \begin{bmatrix}
		\pderiv{\vartheta}{\xi} & \pderiv{\vartheta}{\eta}\\
		\pderiv{\varphi}{\xi}	 & \pderiv{\varphi}{\eta}
	\end{bmatrix} = \begin{bmatrix}
		\pderiv{\vartheta}{x} & \pderiv{\vartheta}{y} & \pderiv{\vartheta}{z}\\
		\pderiv{\varphi}{x} & \pderiv{\varphi}{y} & \pderiv{\varphi}{z}
	\end{bmatrix}\begin{bmatrix}
		\pderiv{x}{\xi} & \pderiv{x}{\eta}\\
		\pderiv{y}{\xi} & \pderiv{y}{\eta}\\
		\pderiv{z}{\xi} & \pderiv{z}{\eta}
	\end{bmatrix}
\end{equation*}
and the derivatives of the basis functions may be computed by
\begin{equation*}
	\begin{bmatrix}
		\pderiv{R_A}{\vartheta}\\
		\pderiv{R_A}{\varphi}
	\end{bmatrix} = J_3^{-\transpose}\begin{bmatrix}
		\pderiv{R_A}{\xi}\\
		\pderiv{R_A}{\eta}	
	\end{bmatrix}.
\end{equation*}