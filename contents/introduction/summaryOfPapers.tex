%\newpage
\section{Summary of Papers}

\subsection{Paper I: Exact 3D scattering solutions for spherical symmetric scatterers}
The first paper sets the stage for proper code development for acoustic scattering problems as it presents a very general exact solution for spherical symmetric scatterers. These scatterers can consist of an arbitrary number of (concentric) elastic shells, each shell with its own material parameters (outer and inner radii, mass density, Youngs modulus and Poisson ratio). Correspondingly, the intermediate fluid layers also have its own material parameters (mass density and speed of sound). Moreover, in addition to the full acoustic structure interaction boundary condition three other boundary conditions (BC) are implemented including sound hard BC (rigid scattering), sound soft BC and a boundary condition modeling an elastic sphere. All in all, the implemented program (in \MATLAB) enables a vast range of benchmark tests for acoustic structure interaction problems. 

Several numerical examples are presented including verification to existing benchmark examples. Numerical experiments of the residual error of the exact solution are also performed showing that the error only originates from round-off errors. Finally, two acoustic structure interaction problems in the time domain are presented using a Fourier transform.

\subsection{Paper II: Isogeometric Analysis of Acoustic Scattering using Infinite Elements}
The second paper investigates the contribution of isogeometric analysis (IGA) to acoustic structure interaction problems with the infinite element method. Four infinite element formulations are investigated including the conjugate and unconjugated versions of both the Petrov--Galerkin formulation and the Bubnov--Galerkin formulation. 

Several numerical experiments on a spherical shell are performed and compared to the exact solution presented in the first paper. In this test setting, the power of the $k$-refinement strategy, unique to IGA, is clearly demonstrated. For resolved meshes $k$-refinement leads to significantly higher accuracy with only a small increase in the number of degrees of freedom. 

A comparison between $C^0$ finite element method (FEM) and IGA is also performed. The exact geometry is of less importance in comparison with higher order FEM, as the increase of inter-element continuity plays a much more significant role in the accuracy improvement as a function of the number of degrees of freedom. 

Finally, the performance of IGA with the infinite element using a prolate ellipsoidal coordinate system is investigated on a stripped BeTSSi submarine model.

\subsection{Paper III: Isogeometric boundary element method}
The third paper continues the active research on isogeometric boundary element method (IGABEM), where the focus is to analyze the approximability of IGABEM and application to complex geometries. 

The usage of the infinite element method for handling the unbounded domain in the second paper is by no means an obvious choice in acoustic scattering problems. Indeed, many good candidates exist, and the boundary element method (BEM) is such an option. The BEM is in particular intriguing in combination with IGA as the bridging of CAD and analysis is improved; there is no need to mesh the geometry between the scatterer and an artificial boundary since BEM only require surface representation. 

The approach in the numerical experiments taken in the second paper is also used in this paper, as it establishes convincing results where analytic solutions exist before moving onto more complex cases. Since BEM only require surface representation, it is much easier to investigate complex geometries. And the outer hull of the BeTSSi submarine is in this regard investigated in depth and compared to independent reference solutions. 

The main conclusions from the usage of IGA in combination with the infinite element method in the second paper also holds true for IGABEM. Moreover, proper numerical integration of weakly singular integrals is investigated and shown to be crucial for obtaining results close to the best approximation. Both the collocation and Galerkin approach are used in combination with several boundary integral equations (BIEs) including the conventional BIE (CBIE) and the Burton-Miller formulation. The application of the latter formulation removes fictitious eigenfrequencies with the cost of somewhat reduced accuracy compared to the CBIE formulation. The application of IGABEM to the BeTSSi submarine posed some problems for non-Lipschitz areas in the model but results with engineering precision ($<1\%$ error) are still obtained.

\subsection{Paper IV: Isogeometric Kirchhoff approximation using numerical steepest descent}
The oscillatory nature of acoustic scattering problems is an intrinsic problem for classical finite element technologies. The usual rule of thumb is a requirement of 10-12 degrees of freedom per wavelength. For high frequencies this requirement becomes too restrictive, especially for 3D scattering problems. This problem is also present for IGA even though it uses slightly less degrees of freedom per wavelength. As a reference, requiring engineering precision at $\SI{30}{kHz}$ for the BeTSSi submarine would require an estimate of \num{27e9} dofs of 3D IGA elements. Which is more than a factor thousands of the computational limit of the clusters available to the author both in terms of memory and computational time. Clearly another approach is needed. 

The idea of enriching the basis functions with the same oscillatory nature as the given problem has shown promising results~\cite{Chandler_Wilde2012nab,Peake2013eib, Peake2014eai,Peake2015eib}. A big challenge remaining on this front is the numerical integration (for 3D problems) as classical quadrature also becomes too computationally expensive. As a step in the direction of solving this problem, the fourth paper considers the simpler\footnote{Although it should be mentioned that the problem of integrating around the shadow boundary with the numerical steepest descent is not present for the boundary element method.} problem of using the numerical steepest descent to evaluate the integrals of the Kirchhoff approximation method. 

The Kirchhoff approximation method is a high frequency approximation where the solution at the boundary is approximated using a physical optics approximation. The classical way of using this method is to tessellate the model into triangles which enables exact integration. However, the number of required triangles is not independent of the frequency, and results in high memory consumption at high frequencies. This problems are avoided entirely with the proposed isogeometric Kirchhoff approach where the boundary data are calculated on the exact CAD model.