\chapter*{Preface}
\addcontentsline{toc}{chapter}{Preface}
\markboth{Preface}{Preface}%
%\Blindtext[4][1]%
The phrase ``A picture is worth a thousand words'' is an adage that resonate throughout this thesis and my academic work. Creating properly made figures to explain scientific work is both crucial and time consuming, but nevertheless very satisfactory in the end. Several opinions exist for how to write good articles, especially when it comes to the length. In the spirit of Thomas Hughes quote ``Don't tell me; show me'', this work contains a lot of figures. Moreover, much of the work is self-contained, resulting in an even longer thesis. Not only does it ease the workflow for the reader, but it also makes the work more accessible to a greater audience.

I was introduced to isogeometric analysis (IGA) by my main supervisor Trond Kvamsdal in the first year of my master's degree, right after an introductory course in the finite element method. My co-supervisor Trond Jenserud proposed to combine IGA with acoustic scattering in my master's thesis entitled ``Isogeometric Analysis of Acoustic Scattering''. This work set the stage for my continued work on the topic for this PhD thesis.

Countless hours have been used programming in \MATLAB resulting in a comprehensive toolbox I have named ASIGA (Acoustic Scattering using IsoGeometric Analysis). I'm an advocate for open research, and for this reason the ASIGA toolbox is open source. The toolbox is far from complete, but it will most likely be improved in future projects.

I would like to thank Trond Kvamsdal for guiding me through every stage of my career as a research scientist. I am not only grateful for Kvamsdal's scientific supervision, but also for giving me so many great life experiences abroad. 
I would also like to thank Trond Jenserud for his supervision and feedback on my work. I have very much enjoyed my stay at IMF\footnote{Department of Mathematical Sciences.}, both as a student and as a PhD-candidate. Moreover, I would like to thank FFI\footnote{Norwegian Defence Research Establishment.} for the support and the opportunity to work on the topic of acoustics in two summer jobs, the second of which (supervised by Jenserud) gave me a head start for my PhD. 
Thanks to Karl Thomas Hjelmervik not only for supervising my first summer job at FFI and his interest for my work, but also for including me into the research community with countless hours of board games. 
Thanks also to Kjetil Andr{\'{e}} Johannessen for his scientific supervision and inspiration throughout the last 6 years.
I would also like to thank Torbj{\o}rn Ringholm, S{\o}lve Eidnes, Hallvard Norheim B{\o}, Morten Andreas Nome, Charles Curry, Winston Heap, Petter Kjeverud Nyland, Tale Bakken Ulfsby, H{\aa}vard Bakke Bjerkevik, Fredrik Arbo H{\o}eg, Ingeborg Gullikstad Hem, Fredrik Hildrum, Kristoffer Varholm, Torstein Fjeldstad, Thea Roksv{\aa}g, Sondre Tesdal Galtung, Nicky Cordua Mattsson, Vanje Rebni Kjer, Maria Lie Selle, Mathias Nikolai Arnesen, Erik Rybakken, Alexander Sigurdsson, Ola Isaac H{\o}g{\aa}sen M{\ae}hlen, and many others for their support and entertainment making my PhD-study truly enjoyable.
Finally, I would like to thank my family for their support and patience throughout this work.