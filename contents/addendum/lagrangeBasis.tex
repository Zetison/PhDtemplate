\section{Lagrange basis functions and GLL nodes}
The Lagrange polynomials are given by\footnote{Note that we do not use the conventional notation $l_i(x)$ for the Lagrange basis functions due to conflict with the index $l$.}
\begin{equation*}
	l_i(\xi) = \prod_{\substack{j = 1 \\ j\neq i}}^n\frac{\xi-\xi_j}{\xi_i-\xi_j}
\end{equation*}
where $\Xi = \{\xi_i\}_{i=1}^n$ are the interpolation nodes. Note the important interpolatory property of the Lagrange basis functions
\begin{equation}\label{Eq5:LagrangeProperty}
	l_i(\xi_j) = \delta_{ij}
\end{equation}
where $\delta_{ij}$ is the Kronecker delta
\begin{equation*}
	\delta_{ij} = \begin{cases}
		1 & i = j\\
		0 & i \neq j.
	\end{cases}
\end{equation*}
Also note that the derivative of the Lagrange basis functions are given by
\begin{equation*}
	l'_i(\xi) = \sum_{\substack{l = 1\\l\neq i}}^n\frac{1}{\xi_i-\xi_l}\prod_{\substack{j = 1\\  j\neq i,l}}^n\frac{\xi-\xi_j}{\xi_i-\xi_j}.
\end{equation*}
In particular,
\begin{equation}\label{Eq5:derivativesAtNodes}
	l'_i(\xi_i) = \sum_{\substack{l=1\\l\neq i}}^n \frac{1}{\xi_i-\xi_l} \quad
\end{equation}
and
\begin{equation}\label{Eq5:derivativesAtNodes2}
	l_i'(\xi_l) = \frac{1}{\xi_i-\xi_l}\prod_{\substack{j = 1\\  j\neq i,l}}^n\frac{\xi_l-\xi_j}{\xi_i-\xi_j}\qquad l \neq i.
\end{equation}
\begin{definition}\label{Def:nodes}
The Gauss-Lobatto-Legendre (GLL\footnote{These nodes also goes by the name Legendre-Gauss-Lobatto (LGL) nodes.}) nodes, $\{\xi_i\}_{i=1}^n$, are the roots of the completed Lobatto polynomial of degree $n$. That is, they are the solutions to the following equation
	\begin{equation*} %\label{Eq5:CompleteLobatto}
		(1-\xi^2)\legendre'_{n-1}(\xi) = 0
	\end{equation*}
	where $\legendre_{\check{p}}(\xi)$ are the Legendre polynomial of degree $\check{p}$.
\end{definition}
Note that $\xi_1= -1$ and $\xi_n=1$. Inspired by \cite[p. 305-306]{Fichtner2010fsw} we present the following theorem\footnote{The converse theorem was proved in \cite[p. 305-306]{Fichtner2010fsw}.}.
\begin{theorem}
	Let $\{\xi_i\}_{i=1}^n$ be the GLL nodes. Then the Lagrange polynomials defined by these nodes have the property
	\begin{equation*}
		l_i'(\xi_i) = \begin{cases}
			-\frac{n(n-1)}{4} & i=1\\
			\frac{n(n-1)}{4} & i=n\\
			0 & \text{otherwise.}
		\end{cases}
	\end{equation*}
\end{theorem}
\begin{proof}
	Consider the \textit{generating polynomial}
	\begin{equation*}
		\Phi_n(\xi) = \prod_{j=1}^n (\xi-\xi_j) = (\xi-\xi_1)(\xi-\xi_2)\cdots(\xi-\xi_n)
	\end{equation*}
	and note that (using the product rule)
	\begin{equation}\label{Eq5:dPhi}
		\Phi_n'(\xi) = \sum_{l=1}^n \prod_{\substack{j=1\\j\neq l}}^n (\xi-\xi_j)\qquad\Rightarrow\qquad  \Phi_n'(\xi_i) = \prod_{\substack{j=1\\j\neq i}}^n (\xi_i-\xi_j)
	\end{equation}
	and
	\begin{equation*}
		\Phi_n''(\xi) = \sum_{l=1}^n \sum_{m=1}^n \prod_{\substack{j=1\\j\neq l,m}}^n (\xi-\xi_j)\qquad\Rightarrow\qquad  \Phi_n''(\xi_i) = 2\sum_{\substack{l=1\\l\neq i}}^n \prod_{\substack{j=1\\j\neq i,l}}^n (\xi_i-\xi_j).
	\end{equation*}
	Then, by writing the sum of fractions in \Cref{Eq5:derivativesAtNodes} as a single fraction (with a common denominator) we get
	\begin{equation}\label{Eq5:BderivPhi}
		l_i'(\xi_i) = \sum_{\substack{j=1\\j\neq i}}^n \frac{1}{\xi_i-\xi_j} = \frac{\displaystyle\sum_{\substack{l=1\\l\neq i}}^n\prod_{\substack{j=1\\j\neq i,l}}^n (\xi_i-\xi_j)}{\displaystyle\prod_{\substack{j=1\\j\neq i}}^n (\xi_i-\xi_j)} =\frac{\frac12\Phi_n''(\xi_i)}{\Phi_n'(\xi_i)}.
	\end{equation}
	Since $\{\xi_i\}_{i=2}^{n-1}$ are the roots of $\legendre_{n-1}'(\xi)$ we may write
	\begin{equation}\label{Eq5:LegendreDeriv}
		\legendre_{n-1}'(\xi) = C\prod_{j=2}^{n-1}(\xi-\xi_j)
	\end{equation}
	for some non-zero constant $C$, and thus
	\begin{equation}\label{Eq5:LegendrePhiCouple}
		 \left(\xi^2-1\right)\legendre_{n-1}'(\xi) = C\Phi_n(\xi).
	\end{equation}
	We know that the Legendre polynomials $\legendre_{n-1}(\xi)$ solves the Legendre differential equation
	\begin{equation}\label{Eq5:LegendreDiff}
		\deriv{}{\xi}\left[\left(1-\xi^2\right)\deriv{}{\xi}\legendre_{n-1}(\xi)\right] + n(n-1)\legendre_{n-1}(\xi) = 0.
	\end{equation}
	Combining \Cref{Eq5:LegendrePhiCouple} and \Cref{Eq5:LegendreDiff} yields
	\begin{equation*}
		-C\Phi_n'(\xi) = -2\xi \legendre_{n-1}'(\xi) + \left(1-\xi^2\right)\legendre_{n-1}''(\xi) = -n(n-1)\legendre_{n-1}(\xi)
	\end{equation*}
	and then
	\begin{equation}\label{Eq5:LegendrePhiCouple2}
		C\Phi_n''(\xi) = n(n-1)\legendre_{n-1}'(\xi).
	\end{equation}
	This means that $\Phi_n''(\xi_i)=0$ for $i=2,\dots,n-1$, and thus, from \Cref{Eq5:BderivPhi} we see that $l_i'(\xi_i)=0$ for $i=2,\dots,n-1$.
	
	Using \Cref{Eq5:dPhi,Eq5:LegendreDeriv,Eq5:LegendrePhiCouple2} we have
	\begin{align*}
		C\Phi_n''(\xi_1) &= n(n-1)\legendre_{n-1}'(\xi_1) = n(n-1)C\prod_{j=2}^{n-1} (\xi_1-\xi_j) \\
		&= C\frac{n(n-1)}{\xi_1-\xi_n} \Phi_n'(\xi_1)  = -C\frac{n(n-1)}{2} \Phi_n'(\xi_1) 
	\end{align*}
	and correspondingly
	\begin{equation*}
		C\Phi_n''(\xi_n) = C\frac{n(n-1)}{2} \Phi_n'(\xi_n) 
	\end{equation*}
	which enables us to write $l_1'(\xi_1)$ and $l_n'(\xi_n)$ as (using \Cref{Eq5:BderivPhi})
	\begin{equation*}
		l_1'(\xi_1) =- \frac{n(n-1)}{4}\quad\text{and}\quad l_n'(\xi_n) = \frac{n(n-1)}{4},
	\end{equation*}
	respectively.
\end{proof}
\begin{remark}
	Using \Cref{Eq5:dPhi} one can rewrite \Cref{Eq5:derivativesAtNodes2} as
	\begin{equation*}
	l_i'(\xi_l) = \frac{1}{\xi_i-\xi_l}\prod_{\substack{j = 1\\  j\neq i,l}}^n\frac{\xi_l-\xi_j}{\xi_i-\xi_j}= \frac{1}{\xi_l-\xi_i}\frac{\Phi_n'(\xi_l)}{\Phi_n'(\xi_i)}\qquad l \neq i.
	\end{equation*}
\end{remark}
\begin{remark}
	Using \Cref{Eq5:derivativesAtNodes2} one can evaluate the following special cases in closed form
	\begin{equation*}
	l_n'(\xi_1) = \frac{(-1)^n}{2}\quad\text{and}\quad l_1'(\xi_n) = -\frac{(-1)^n}{2}.
	\end{equation*}
\end{remark}