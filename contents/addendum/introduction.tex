\section{Introduction}
Acoustic scattering problems on unbounded domain has been tackled with a vast set of methods. For the finite element method in this context, several methods has been used to solve the problem of unboundedness of the domain, including the method of perfectly matched layer (PML) \cite{Bermudez2007aop,Michler2007itp}, local differential absorbing boundary condition operators \cite{Shirron1995soe,Bayliss1982bcf,Hagstrom1998afo,Tezaur2001tdf}, Dirichlet to Neumann operators \cite{Givoli2013nmf} and infinite elements \cite{Ihlenburg1998fea,Bettess1977ie,Bettess1977dar}. The boundary element method has also been investigated thoroughly  \cite{Sauter2011bem,Schanz2007bea,Marburg2008cao,Chandler_Wilde2012nab}. Moreover, high frequency approximation methods like the Kirchhoff approximation method \cite{Foote2002cka,Fillinger2014aen} and ray/beam tracing \cite{Jensen2011coa,Burgschweiger2014rot,Heckbert1984btp} has been of particular interest for the high frequency spectrum as the classical element methods becomes too computationally expensive in this realm. An attempt to solve this problem for element methods in acoustics has been suggested in \cite{Chandler_Wilde2012nab,Peake2013eib,Peake2015eib} where the basis functions are enriched by oscillatory functions.

Away from non-smooth parts of the scatterer, the solution is smooth, and it would make sense to exploit this fact by using smooth basis functions in the analysis. As illustrated in \cite{Venas2018iao}, using IGA with $C^{\check{p}-1}$ continuity instead of the classical FEM $C^0$ continuity increases the accuracy. The natural question would then be if even higher continuity continues to improve the results. The method of fundamental solution \cite{Fairweather2003tmo} is such an attempt where the $C^\infty$ basis functions are collocated at the boundary to satisfy the boundary condition. This method has spectral convergence for smooth geometries. However, as these basis functions do not incorporate the reduced continuity at the non-smooth parts of the scatterer, sub-optimal results would be obtained in these cases. A remedy for this would be to use the spectral element method where one naturally incorporates the $G^0$ continuity of the scatterer into the basis functions of $C^0$ continuity across the element boundaries and maintains the $C^\infty$ continuity in the interior of each element. Surprisingly little work has been done using spectral element methods for solving acoustic scattering problems. Mehdizadeh et al. \cite{Mehdizadeh2003ioa} investigated the 2D spectral element method using the PML method and J{\o}rgensson et al. \cite{Jorgensson2018ras} used spectral element methods for 2D room acoustics. The lacking literature on the usage of spectral element analysis of 3D scattering problems using infinite elements is therefore a motivating factor of this work.